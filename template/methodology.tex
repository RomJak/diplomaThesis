\chapter{Methodology}

In this chapter the methodology including the used concepts such as survey for user segmentation, paper prototyping is described. Firstly, the approach is outlined, followed by the use-case description. Then, the conceptional architecture is explained, which is described in more detail in the Design Methods section.

\section{User segmentation}




\section{Elicitation of Requirements with Paper Prototyping}
For the Paper Prototype a Step-by-Step guide proposed by Arnowitz et al. \cite{arnowitz2010effective} was used. To create a Paper Prototype the following steps should be done:

\begin{enumerate}
	\item \textbf{Create scenario}. Before starting to draw anything the main user goals and tasks have to be portrayed. This can be done in a scenario narration.
	\item \textbf{Inventory UI elements}. The next step is to make a checklist of all UI elements that may be needed to support the scenario.
	\item \textbf{Create UI elements}. All the UI elements from the checklist from the previous step are now created in paper form. There are a lot of tools and materials that can come in handy at this step. The following list of materials might help the process: paper, sticky notes,
	whiteboard, sketchbook, notebook, napkin, cards, overhead sheet, cardboard, carton, scissors, markers, UI stencil, correction fluid and tape and transparency sheet. 
	\item \textbf{Run through scenario}. In this step a dry-run through the scenario with the paper prototype should be done and missing parts should be found an recreated.
	\item \textbf{Internal review}. The last step in the first round is the internal review with the team where the audience is defined, the goals for each version of the prototype is reviewed, the expectation of the reviewers are found out and the next steps are planned.
\end{enumerate}

The next Step-By-Step Guide is following the first. It was also proposed by Arnowitz et al. \cite{arnowitz2010effective} and is for testing the Paper Prototype:

\begin{enumerate}
	\item \textbf{Revise scenario}. The internal review may have uncovered some tweaks that you want to change. Be careful for changes at the scenario as it may cause a ripple effect which can lead to necessary changes in user interface elements or even new screens. Keeping changes to minimum is recommended. If changes are necessary keep in mind that this may result in the inventory of user interface.
	\item \textbf{Revise inventory UI elements}.
	Until now maybe multiple run throughs through the prototype have been done and you noticed that some vital pieces of the interface are missing. Now is a good time to check completeness of the UI elements checklist. Developing a set of UI elements for cases that you did not anticipate may be also useful.
	\item \textbf{Create UI elements}.
	
	
	\item \textbf{Pilot run through scenario}. Before presenting the prototype to the user it has to be tried out first. You can give the Prototype to anyone, e.g. a team member, to try it out. The aim here is to find missing pieces to be prepared for everything they do. The run through will ensure that you haven't created a half-baked prototype.
	
	\item \textbf{Internal review}. 
	
	The last step in the first round is the internal review with the team where the audience is defined, the goals for each version of the prototype is reviewed, the expectation of the reviewers are found out and the next steps are planned.
	
	
	\item \textbf{Prepare Kit}. Before running the prototype session the papers have to be arranged in a way that makes it easy to find the various UI elements. Also blank paper, sticky notes and pens should be prepared for further ideas.
	\item \textbf{The Prototype Session}. The user study session is an interactive process where one ore more participants and a facilitator are involved. In a dialogue the participant completes tasks provided by the facilitator. The session is used to get user opinions about early design and task flow ideas represented on paper. The sessions are typically recorded for later examination. The feedback from the users show what they expect from the app which is of great value for the implementation later on \cite{snyder2003paper}. Weiss  \cite[p.~144]{weiss2003handheld} proposes to invite not only one, but two respondents at a time for paper prototype usability tests. He mentions that two respondents feel more comfortable in the casual environment that paper prototyping creates, whereas one single respondent can easily become overwhelmed by the experience.
	\item \textbf{Reiterate}. After each prototype session an review and evaluation about what went good and what bad can be done. Although it might be tempting to change things after each session, it is better to wait until all the planned user sessions are done to do an overall comparative review at the end.
\end{enumerate}

\section{Conceptual Design}