% Copyright (C) 2014-2016 by Thomas Auzinger <thomas@auzinger.name>

\documentclass[draft,final]{vutinfth} % Remove option 'final' to obtain debug information.

% Load packages to allow in- and output of non-ASCII characters.
\usepackage{lmodern}        % Use an extension of the original Computer Modern font to minimize the use of bitmapped letters.
\usepackage[T1]{fontenc}    % Determines font encoding of the output. Font packages have to be included before this line.
\usepackage[utf8]{inputenc} % Determines encoding of the input. All input files have to use UTF8 encoding.

% Extended LaTeX functionality is enables by including packages with \usepackage{...}.
\usepackage{amsmath}    % Extended typesetting of mathematical expression.
\usepackage{amssymb}    % Provides a multitude of mathematical symbols.
\usepackage{mathtools}  % Further extensions of mathematical typesetting.
\usepackage{microtype}  % Small-scale typographic enhancements.
\usepackage[inline]{enumitem} % User control over the layout of lists (itemize, enumerate, description).
\usepackage{multirow}   % Allows table elements to span several rows.
\usepackage{booktabs}   % Improves the typesettings of tables.
\usepackage{subcaption} % Allows the use of subfigures and enables their referencing.
\usepackage[ruled,linesnumbered,algochapter]{algorithm2e} % Enables the writing of pseudo code.
\usepackage[usenames,dvipsnames,table]{xcolor} % Allows the definition and use of colors. This package has to be included before tikz.
\usepackage{nag}       % Issues warnings when best practices in writing LaTeX documents are violated.
\usepackage{todonotes} % Provides tooltip-like todo notes.
\usepackage{hyperref}  % Enables cross linking in the electronic document version. This package has to be included second to last.
\usepackage[acronym,toc]{glossaries} % Enables the generation of glossaries and lists fo acronyms. This package has to be included last.

% Define convenience functions to use the author name and the thesis title in the PDF document properties.
\newcommand{\authorname}{Romana Jakob} % The author name without titles.
\newcommand{\thesistitle}{An Usability Design Approach of Tailored
	Visualizations for Mobile Applications} % The title of the thesis. The English version should be used, if it exists.

% Set PDF document properties
\hypersetup{
    pdfpagelayout   = TwoPageRight,           % How the document is shown in PDF viewers (optional).
    linkbordercolor = {Melon},                % The color of the borders of boxes around crosslinks (optional).
    pdfauthor       = {\authorname},          % The author's name in the document properties (optional).
    pdftitle        = {\thesistitle},         % The document's title in the document properties (optional).
    pdfsubject      = {Subject},              % The document's subject in the document properties (optional).
    pdfkeywords     = {a, list, of, keywords} % The document's keywords in the document properties (optional).
}

\setpnumwidth{2.5em}        % Avoid overfull hboxes in the table of contents (see memoir manual).
\setsecnumdepth{subsection} % Enumerate subsections.

\nonzeroparskip             % Create space between paragraphs (optional).
\setlength{\parindent}{0pt} % Remove paragraph identation (optional).

\makeindex      % Use an optional index.
\makeglossaries % Use an optional glossary.
%\glstocfalse   % Remove the glossaries from the table of contents.

% Set persons with 4 arguments:
%  {title before name}{name}{title after name}{gender}
%  where both titles are optional (i.e. can be given as empty brackets {}).
\setauthor{}{\authorname}{BSc.}{female}
\setadvisor{Ao.Univ. Prof. Mag. Dr.}{Margit Pohl}{}{female}

% For bachelor and master theses:
\setfirstassistant{Mag. Dr.}{Gerhard Engelbrecht}{}{male}


% For dissertations:
\setfirstreviewer{Pretitle}{Forename Surname}{Posttitle}{male}
\setsecondreviewer{Pretitle}{Forename Surname}{Posttitle}{male}

% For dissertations at the PhD School and optionally for dissertations:
\setsecondadvisor{Pretitle}{Forename Surname}{Posttitle}{male} % Comment to remove.

% Required data.
\setaddress{Nabegg 53/2, 3323 Neustadtl}
\setregnumber{1227095}
\setdate{01}{05}{2018} % Set date with 3 arguments: {day}{month}{year}.
\settitle{\thesistitle}{An Usability Design Approach of Tailored
	Visualizations for Mobile Applications} % Sets English and German version of the title (both can be English or German).
\setsubtitle{}{} % Sets English and German version of the subtitle (both can be English or German).

% Select the thesis type: bachelor / master / doctor / phd-school.
% Bachelor:
% \setthesis{bachelor}
%
% Master:
\setthesis{master}
\setmasterdegree{dipl.} % dipl. / rer.nat. / rer.soc.oec. / master
%
% Doctor:
%\setthesis{doctor}
%\setdoctordegree{rer.soc.oec.}% rer.nat. / techn. / rer.soc.oec.
%
% Doctor at the PhD School
%\setthesis{phd-school} % Deactivate non-English title pages (see below)

% For bachelor and master:
\setcurriculum{Business Informatics}{Wirtschaftsinformatik} % Sets the English and German name of the curriculum.

% For dissertations at the PhD School:
\setfirstreviewerdata{Affiliation, Country}
\setsecondreviewerdata{Affiliation, Country}


\begin{document}

\frontmatter % Switches to roman numbering.
% The structure of the thesis has to conform to
%  http://www.informatik.tuwien.ac.at/dekanat

\addtitlepage{naustrian} % German title page (not for dissertations at the PhD School).
\addtitlepage{english} % English title page.
\addstatementpage

%\begin{danksagung*}
%Ich möchte jedem danken, der mir in irgendeiner Weise beim Verfassen dieser Arbeit geholfen hat. Besonders möchte ich meiner Betreuerin Margit Pohl für Ihre Unterstützung danken und dass Sie mich trotz längerer Pausen nicht aufgegeben hat.
%
%Vielen Dank an Gerhard Engelbrecht, der mit seinen Anregungen viel zur Entstehung dieser Arbeit beigetragen hat und es möglich gemacht hat, diese Arbeit in Zusammenarbeit mit Siemens zu schreiben. Danke auch an meinen Chef bei Siemens, Herwig Schreiner, der mir die Freiheit gegeben hat, diese Arbeit zu schreiben.
%
%Danke an alle, die sich die Zeit genommen haben, um die in dieser Arbeit entstandene mobile Applikation zu testen. Euer Feedback war sehr wertvoll und hat viel zur weiteren Entstehung beigetragen.
%
%Schließlich danke ich meinen Freunden während der Studienzeit für die gegenseitige Unterstützung.
%\end{danksagung*}

\begin{acknowledgements*}
I want to thank everyone who has contributed in any way to my study and the making of this thesis. First, I want to thank Margit Pohl, my thesis advisor for her support and not giving up on me, despite longer breaks. \\
I am grateful to Gerhard Engelbrecht not only for his feedback but also for the opportunity to write the thesis in cooperation with Siemens. Thanks to my boss at Siemens, Herwig Schreiner, for giving me the freedom to pursue this thesis.\\
Thanks to everyone who spared time to test the mobile application which was developed within this thesis. Your feedback was very helpful and contributed a lot to its further development. Finally, thanks to my friends during my study time for the mutual support.


\end{acknowledgements*}

\begin{kurzfassung}
\todo{Ihr Text hier.}
\end{kurzfassung}

\begin{abstract}
\todo{Enter your text here.}
\end{abstract}

% Select the language of the thesis, e.g., english or naustrian.
\selectlanguage{english}

% Add a table of contents (toc).
\tableofcontents % Starred version, i.e., \tableofcontents*, removes the self-entry.

% Switch to arabic numbering and start the enumeration of chapters in the table of content.
\mainmatter

% Remove following line for the final thesis.
%\input{intro.tex} % A short introduction to LaTeX.

% CHAPTER Introduction
\chapter{Introduction}

In the age of social media, where information is tailored to users' interests, preferences and state of education, the question arises how to integrate this phenomenon into common mobile applications. Especially when it comes to education and behaviour change an adaption of the user interface to various requirements might be useful.

This thesis investigates whether tailoring the interface of a mobile application to different needs is useful. The needs are gathered into user groups, in order to limit the amount of possibilities.


\section{Motivating Scenario}

This thesis is written in cooperation with Siemens AG Austria, within the research project that deals with the Seestadt in Aspern. The Seestadt is one of the biggest city development projects in Europe \footnote{https://www.aspern-seestadt.at/ Accessed 10.01.2018}. The Aspern Smart City Research GmbH \& Co KG \footnote{http://www.ascr.at/ Accessed 10.01.2018} (ASCR) is an exclusive technology partner of Siemens AG. The Aspern project has the overall goal of finding smarter solutions for energy consumption with the help of smart grids, power supplies, building systems, intelligent power grids, and information
and communication technologies (ICT) interacting in an optimal manner. The ASCR infrastructure manages the data coming from smart grids and smart buildings such as temperature, energy consumption, water consumption, power demand as well as external data sources such as weather, city events, energy market, traffic reports etc. \cite{parreira2015role} In total 1.5 million values are measured per day. The create something useful out of this amount of big data is a big task. \footnote{http://www.report.at/index.php/energie/wirtschaft-a-politik/item/91884-lebendes-stadtlabor Accessed 10.01.2018}

Take, for example, an application that informs you about your electricity consumption. What can be assumed, is that the user wants an easy-to-use application which shows the power consumption in an understandable way. The problem that we observed is that the majority of users lack the feeling for the size of one kilowatt hour. The same can be witnessed when it comes to CO2 emission. The unit of kilograms of CO2 is an information that mostly only experts can grasp and can relate to.

In the field of software development the interaction with the user is important, including the consideration of a user's knowledge. Numerous applications aim at motivating the user to save energy or CO2 but neglect the incomprehensibility of units of measurements one does not deal with on a daily basis. The sense of trying to motivate the user to save energy by displaying the electricity consumption in kilowatt hours, might have less impact than setting it at least in relation to an average consumption of electricity or even visualizing it with a gamification approach. On the other hand, for someone who is easy on these types of measurement a visualization with colours or graphs might be too much.

\section{Problem Statement}

So the problem we, this thesis is written in cooperation with Siemens AG Austria, are facing is to develop a mobile application that is beneficial for all types of users, starting from users who do not have a feeling for kilowatt hours or kilograms of CO2 up to users having a great affinity for electricity and carbon-dioxide emission.

To address this bandwidth of user knowledge and visualization possibilities, this thesis investigates the usefulness of tailoring a mobile application to a users knowledge. Furthermore design principles and criteria that shall help front-end developer, usability engineers as well as software architects to develop applications customized to a users level of knowledge shall be investigated.

We evaluate different types of users and their preferred way of gathering information. Ranging from the ones who show only interest in their overall behaviour, meaning if they are better or worse than the average, over others, who want to know their power consumption more detailed but still can't grasp the measurement of one kilowatt hour, up to users, who are deep into the topic and are keen on extensive figures.

\section{Aim of the work} 
The overall goal of this thesis is to identify the benefits or even drawbacks of providing a user interface in a mobile application with various possibilities of presenting information to switch between. We want to investigate if a user makes use of different visualizations or the presented way is excepted and therefore an adaptation of the user to the application takes place.\\
This thesis contributes 
(1) a prototype of a mobile application aiming at increasing CO2 awareness with the help of customized visualizations and
(2) a catalogue of criteria of design principles for tailoring visualizations containing information of consumption data.
This thesis aims at finding an answer to the following central research questions:

\textbf{What are the effects on knowledge acquisition in mobile applications when providing visualizations tailored to users' knowledge?}
\\\\
The central research question can be answered after having found a solution to the sub-questions:\\\\
\textbf{(a) What are the characteristics of a user group with the same state of knowledge?}
In order to answer this research question we first conduct a literature review in the area, followed by a user survey detecting the state of knowledge in the field of electricity units of measurements, i.e. the size of one kilowatt hour, one kilogram of CO2. These findings will help in identifying groups and their characteristics.\\\\
\textbf{(b) Which criteria do questions have to meet, that shall identify the type of a user?}
The findings of the sub-research question (a) will have an influence on the questionnaire needed for defining which group a user can be assigned to. This questionnaire will be the first contact point in the mobile application.
\\\\
\textbf{(c) What are the design possibilities when it comes to tailoring interfaces to a users' state of knowledge in the scope of electricity consumption data?}
This question can be answered by conducting a literature review and considering the characteristics of a user group.
\\\\
\textbf{(d) Do the characteristics of user groups correlate with the users' preferred type of visualization?}
The results elicited for research question a) are the foundation for defining the correlation between groups of users and their preferred type of visualization. Assuming the favourite type of visualization is the most used one, allows to identify the preferred type of visualization by analysing the log files.
\\\\
\textbf{(e) Does a user switch between various screens showing the same information represented in different ways?}
We answer this by looking at the log files and also by observing the interaction with the mobile application in the usability tests.

%---------------------------------------------------------------------------
\section{Methodological Approach}
%---------------------------------------------------------------------------

In order to fulfill the research questions the methodological approach comprises the following steps:
\begin{enumerate}
	\item \textbf{Literature Review} \\
	The first step is to dive into the topic of usability engineering, especially different forms of visualizations and graphical user interfaces in the scope of mobile applications. That implies a research about paper prototyping, usability testing in the mobile context as well as user classification and carbon dioxide awareness. The goal is to get an insight of all relevant aspects which will serve as foundation for the following steps.
	
	\item \textbf{Comparative analysis of alternatives and comparison of existing approaches} \\
	In this step, the market and competition analysis which was done when the problem arose will be done in more depth. The questions that shall be answered in this steps are the following.
	\begin{itemize}
		\item Which applications are there within the topics of energy saving and CO2 awareness?
		\item Which approaches and visualizations do these applications make use of to increase awareness?
		\item How do these applications handle the users' level of education concerning energy units of measurements, such as kWh?
	\end{itemize}
	
	\item \textbf{Elicitation of requirements with Paper Prototyping} \\
	The second step is to do "Paper Prototyping" in order to elicit the requirements for the graphical user interfaces and overall for the CO2 awareness app. According to \cite{lancaster2004paper} the numerous benefits of early usability studies are vastly superior. It may seem low-tech, but conducting usability tests at this step show what users really expect on a quite detailed level which gives maximum feedback for minimum effort \cite{weiss2003handheld}.
	
	At first a group of people containing at least one user for each user type will be put together.  Next, hand-sketched drafts will be drawn, showing the app with menus, dialog boxes, notifications and buttons. Then, different tasks that can be done with the app shall be defined. These tasks are then conducted by the users. The feedback from the users show what they expect from the app which is of great value for the implementation later on \cite{snyder2003paper}.
	
	\item \textbf{Architectural Design of the CO2 awareness mobile application} \\
	The insights from the previous steps will influence the architecture and the designs of the CO2 awareness mobile application. With focus on design and usability an architectural design will be developed including a development plan. At this step, the different data resources for the computation of the personal CO2 emission, such as power consumption, water consumption, nutrition lifestyle, transportation habits, size of the living space, place of living, family situation etc. must be considered.
	
	The app shall be usable for all users but will be particularly useful for inhabitants of the Aspern Seestadt in Vienna, as we have a database for the dwellers of the student dorm, the schoolhouse and one residential building. This data comes from the Aspern Smart City Research\footnote{http://www.ascr.at/. Accessed 9.11.2017} (ASCR) project where Siemens plays an essential role in collaboration research. 
	
	\item \textbf{Technical Implementation of the CO2 awareness mobile application} \\
	According to the architecture description from step 4 the mobile application will be implemented using an agile software development process and a fully native approach targeting Android Devices.
	
	\item \textbf{Usability Tests} \\
	In order to avoid distorting of the research results the graphical user interface will be tested empirically with 4-5 usability tests, that means the usability is accessed by testing the interface with real users \cite{nielsen1994usability}.
	
	\item \textbf{User study} \\
	The design of the user study will follow the seminal guidelines for conducting case study research in software engineering as proposed by Runeson et al. \cite{runeson2012case}. The target group will consist of at least one user for each type of energy user. The study protocol will follow the check-lists for reading and reviewing case studies from H\"ost and Runeson \cite{host2007checklists}.
	
	\item \textbf{Evaluation} \\
	In this step the developed mobile app will be empirically evaluated against a valuation model in a user study to identify the success of the research. The evaluation model comprises of numerous Key Performance Indicators (KPIs). An extraction of these KPIs is listed in the following:
	\begin{enumerate}
		\item More than 50 \% of all the users using the app state that the possibility of switching between different ways to display the information is useful
		\item More than 50 \% state, that they are more aware of what to do to avoid CO2 than before using the app
		\item More than 50 \% of the users state that they understand and get a feeling of how much CO2 they are producing
	\end{enumerate}
	
	
\end{enumerate}


\section{Structure of the work} 
The remainder of this thesis is structured as follows:  Chapter 2 provides an overview of related work where the main approaches of tailoring user interfaces are discussed. This chapter is concluded by a comparison with the existing approaches.

Subsequently in Chapter 3 the methodology is presented, where an overall architecture is explained apart from the concrete implementation. This is followed by an example use case, motivating the implementation of this thesis. Then, the conceptional architecture is explained, which is described in more detail in the Data Models and Design Methods section.

Afterwards, in Chapter 4 the main work of this thesis, the implementation, is presented. Within this chapter implementation-specific details are discussed.

Then, Chapter 5 evaluates the implementation in form of an example work flow and presents all the opportunities within the framework.

Chapter 6 critically reflects and compares the implementation with related work and discusses open issues.

This thesis is concluded in Chapter 7 with a summary and outlook on future work.



% CHAPTER State of the art
%---------------------------------------------------------------------------
\chapter{State of the Art}
%---------------------------------------------------------------------------

In the following sections the theoretical background for the topics that this thesis deals with will be presented. In particular, usability engineering especially different forms of visualizations and graphical user interfaces in the scope of mobile applications. That implies a research about paper prototyping, usability testing in the mobile context as well as user classification for the definition of user groups and carbon dioxide awareness in general. Finally, we will have a look on existing approaches, such as serious games and a comparative analysis of alternatives.

\section{Usability engineering}

The International Organization for Standardization (ISO) defines usability as the "Extent to which a product can be used by specified users to achieve specified goals with effectiveness, efficiency and satisfaction in a specified context of use" \cite{bevan1998iso}. This definition comprises three measurable attributes which are the following:

\begin{itemize}
	\item \textbf{Effectiveness}: Accuracy and completeness with which users achieve specified goals.
	\item \textbf{Efficiency}: Resources expended in relation to the accuracy and completeness with which users achieve goals.	
	\item \textbf{Satisfaction}: Freedom from discomfort, and positive attitudes towards the use of the product.
\end{itemize}

The ISO standard also identifies three factors that should be considered when evaluating usability:
\begin{itemize}
	\item \textbf{User}: Person who interacts with the product.
	\item \textbf{Goal}: Intended outcome.
	\item \textbf{Context of use}: Users, tasks, equipment (hardware, software and materials), and the physical and social environments in which a product is used.
\end{itemize}

Nielsen \cite{nielsen1994usability} also identified five attributes of usability and factors having an impact on how the user interacts with a system. In addition to the above ones Nielsen \cite{nielsen1994} states:

\begin{itemize}
	
	\item \textbf{Learnability}: The user should get work done rapidly which is possible if the system is easy to use.
	\item \textbf{Efficiency}: Once the user has learned to operate with the system, the productivity should be high.
	\item \textbf{Memorability}: In case a user does not use the system in a longer period, it should, nevertheless, be easy remembered without having to learn everything all over again.	
	\item \textbf{Errors}: When using the system, the user makes few errors and is able to return and recover easily after an error. Further, catastrophic errors must not occur.
	\item \textbf{Satisfaction}: The system is highly accepted as the user has positive attitudes towards the system and finds it pleasant to use.
\end{itemize}


\subsection{Usability engineering in the context of mobile applications}

The focus on usability and interaction between human and hand-held electronic devices has its origin within the emergence of mobile devices. The approach of Nielsen, mentioned above, was expanded with the scope of mobile applications by Zhang and Adipat \cite{zhang2005challenges} who highlighted a number of issues by the advent of mobile devices. The issues mentioned are:
\begin{itemize}
	
	\item mobile context
	\item connectivity
	\item small screen
	\item different display resolution
	\item limited processing capability and power and
	\item data entry methods
	
\end{itemize}

Zhang et al. mention that these restrictions are especially a problem when it comes to usability testing methods, as all these issues must be considered in order to select an appropriate research methodology. It must be kept in mind that contextual factors on perceived usability can occur when they are not considered in a study.

Harisson et al. \cite{harrison2013usability} build up on the terms mentioned before and introduced a PACMAD (People At the Centre of Mobile Application Development) model which was designed to address the limitations of existing usability models when applied to mobile devices. PACMAD extends the theories of usability with more aspects such as \textit{user task} and \textit{context of use}. The existing usability models such as those proposed by Nielsen \cite{nielsen1994usability} and ISO \cite{bevan1998iso} also recognize these factors as crucial parts on which the successfulness of the usability of an application depends. The difference is that PACMAD includes all the factors into one model to ensure a complete usability evaluation.

Deka \cite{deka2016data} discusses how data-driven approaches are tools for mobile app design. A relevant field mentioned is interaction mining, that captures the static part, such as layouts and visual details, as well as the dynamic part, like user flows and motion details, of app design.

Fogarty and Hudson \cite{fogarty2003gadget} presented an experimental toolkit to support optimization for interface and display generation. This approach


The decades of research in adaptive user interfaces were summarized by Gajos et al. \cite{gajos2008decision}. They conclude that personalized user interfaces have the ability to improve user satisfaction and performance, when the interface is adapted to the device, task, preferences and abilities of a person. To automatically generate user interfaces they use decision-theoretic optimization which includes functional specifications of the interface, constraints of the devices e.g. screen size and a list of available interactors, a typical usage trace and a cost function. The cost function holds user preferences and the expected speed of operation.

\section{Elicitation of requirements with Paper Prototyping}
In order to elicit the requirements for the graphical user interfaces and overall for the CO2 awareness app we make use of Paper Prototyping. According to \cite{lancaster2004paper} the numerous benefits of early usability studies are vastly superior. Besides saving time and money by solving problems before the implementation even begins, Paper Prototyping stimulates creativity as it allows to experience with different ideas before committing to one \cite{snyder2003paper}. It may seem low-tech, but conducting usability tests at this step show what users really expect on a quite detailed level which gives maximum feedback for minimum effort \cite{weiss2003handheld}. 

A highly recommended introduction into effective prototyping is provided by Arnowitz et al. \cite{arnowitz2010effective} as well as by Bernard and Summers \cite{bernard2010dynamic} which inducted into dynamic prototyping. Arnowitz et al. \cite{arnowitz2010effective} proposed the following Step-by-Step guide to create a Paper Prototype:

\begin{enumerate}
	\item \textbf{Create scenario}. Before starting to draw anything the main user goals and tasks have to be portrayed. This can be done in a scenario narration.
	\item \textbf{Inventory UI elements}. The next step is to make a checklist of all UI elements that may be needed to support the scenario.
	\item \textbf{Create UI elements}. All the UI elements from the checklist from the previous step shall be created in paper form.
	\item \textbf{Run through scenario}. In this step a dry-run through the scenario with the paper prototype should be done and missing parts should be found an recreated.
	\item \textbf{Internal review}. The last step in the first round is the internal review with the team where the audience is defined, the goals for each version of the prototype is reviewed, the expectation of the reviewers are found out and the next steps are planned.
\end{enumerate}

The next Step-By-Step Guide is following the first. It was also proposed by Arnowitz et al. \cite{arnowitz2010effective} and is for testing the Paper Prototype.

\begin{enumerate}
	\item \textbf{Revise scenario}. The internal review may have uncovered some tweaks that you want to change. Be careful for changes at the scenario as it may cause a ripple effect which can lead to necessary changes in user interface elements or even new screens. Keeping changes to minimum is recommended. If changes are necessary keep in mind that this may result in the inventory of user interface.
	\item \textbf{Revise inventory UI elements}.
	Until now maybe multiple run throughs through the prototype have been done and you noticed that some vital pieces of the interface are missing. Now is a good time to check completeness of the UI elements checklist. Developing a set of UI elements for cases that you did not anticipate may be also useful.
	\item \textbf{Create UI elements}.
	
	
	\item \textbf{Pilot run through scenario}. Before presenting the prototype to the user it has to be tried out first. You can give the Prototype to anyone, e.g. a team member, to try it out. The aim here is to find missing pieces to be prepared for everything they do. The run through will ensure that you haven't created a half-baked prototype.
	
	\item \textbf{Internal review}. 
	
	The last step in the first round is the internal review with the team where the audience is defined, the goals for each version of the prototype is reviewed, the expectation of the reviewers are found out and the next steps are planned.
	
	
	\item \textbf{Prepare Kit}. Before running the prototype session the papers have to be arranged in a way that makes it easy to find the various UI elements. Also blank paper, sticky notes and pens should be prepared for further ideas.
	\item \textbf{The Prototype Session}. The user study session is an interactive process where one ore more participants and a facilitator are involved. In a dialogue the participant completes tasks provided by the facilitator. The session is used to get user opinions about early design and task flow ideas represented on paper. The sessions are typically recorded for later examination. The feedback from the users show what they expect from the app which is of great value for the implementation later on \cite{snyder2003paper}. Weiss  \cite[p.~144]{weiss2003handheld} proposes to invite not only one, but two respondents at a time for paper prototype usability tests. He mentions that two respondents feel more comfortable in the casual environment that paper prototyping creates, whereas one single respondent can easily become overwhelmed by the experience.
	\item \textbf{Reiterate}. After each prototype session an review and evaluation about what went good and what bad can be done. Although it might be tempting to change things after each session, it is better to wait until all the planned user sessions are done to do an overall comparative review at the end.
\end{enumerate}






\section{User Classification}


Weiss \cite{weiss2003handheld} discussed the balance of ease of use compared to the ease of learning. A huge emphasis is laid on the first, and according to Weiss, the most important step in the design and development process, the understanding of the audience. The purpose of the audience definition is to describe the target group, its' traits and ranges.

\subsection{User Segmentation according to Smart Cities Demo Aspern}

Aspern Smart City Research GmbH \& Co KG (ASCR) also lays emphasis on understanding the user. The research group defines a smart user as a person who has the knowledge for sustainable decisions in relation to his or her lifestyle. Saving CO2 and energy should be the overall goals of a smart user.

Nevertheless, not all smart users are the same and not all share the same state of knowledge or interest. Therefore, in 2015 ASCR conducted a socio-scientific study to find out how much know-how a smart user has in the field of technology and energy and also how much interest they have in the topics of energy and sustainability. The research was done in an apartment block named D12, where the possibility to test solutions is given, as the apartments in this block are equipped with systems that collect data including

\begin{itemize}
	\item electricity consumption
	\item room temperature
	\item warm/cold water consumption and
	\item air quality.
\end{itemize}

Over half of the households in the apartment block D12 agreed on making their data available for research purposes and to participate in surveys and workshops. In total, 85 households took part in the study in 2015. In the starting phase two studies were done. At first a qualitative study with personal interviews with selected tenants of the building D12 was done followed by a quantitative study with written questionnaires. One outcome of these studies was the segmentation of users into groups. Different types of users were clustered into four segments according to their state of knowledge and their interest in technology and energy. The user groups also serve as target groups for the development of new technology solutions such as home automation, mobile application and for the development of range of services. The segmentation into groups also makes communication easier as the used methods of communicaiton can be tailored to the needs of a group.

The qualitative study with its interviews was done before the tenants moved into the apartments in Seestadt. Surprisingly the majority stated that it has basic knowledge for the interpretation of the energy consumption and energy data in general. Often they stated that they do not know how much on kilowatt hour is. In most cases the main source of information for energy topics is the energy consumption calculation. Unfortunately the calculation does not state the behaviour or the devices which use up the most energy. Exactly these two aspects are the most wished information for the users when it comes to saving energy.

The aim of a segmentation in its statistical way is to find distinct groups with significant differences \cite{punj1983cluster}. Within a group the characteristics should be homogeneous. An established way for segmentation in statistics is to do two statistic procedures, beginning with a factor analysis, followed by a cluster analysis \cite{tuffery2011data}.

The factor analysis reduces dimensions \cite{williams2010exploratory}. In the quantitative questionnaires multiple variables are collected and in the factor analysis these variables are reduced to so called latent variables or factors. Therefore, the factor analysis shows which dimensions are underlying the whole questionnaire.

In the socio economic study of ASCR an explorative non-rotating factor analysis was calculated. Afterwards the scree test showed the amount of factors, which was in this case four. In terms of content the analysis of the factor showed the following dimensions:

\begin{itemize}
	\item \textbf{Comfort-centered}: This factor covers aspects like home automation, energy relevant user behavior such as lighting and circulation behavior and hot water usage. 
	\item \textbf{Technology-centered}: Also covers aspects like home automation but more with the sense of interest in the technology rather than the comfort aspect.
	\item \textbf{Data sensibility}: Concerns regarding the further use of the collected data.
	\item \textbf{Living in Seestadt}: The aspect of living in the Seestadt as an extra dimension shows that it is some kind of prestige to live there.
\end{itemize}

Finally a cluster analysis was done to identify the user segments. Cluster analysis is an exploratory process with the aim of finding groups of similar objects \cite{tuffery2011data}. Different hierarchical analysis were calculated to find an appropriate amount of clusters. Appropriate means in this case having an big enough group of cases/persons and groups having distinct features. The data set comprised 121 handed back questionnaires and the cluster analysis could identify four clusters. The four clusters correspond to the four user groups. The result of the cluster analysis is shown in ~\ref{fig:cluster} and explained in the following.

\begin{figure}[h]
	\centering
	\begin{tikzpicture}
	\pie [rotate = 180, color = {gray!80, gray!20, gray!40, gray!60}]
	{48/Professionals,  9/Hedonists, 13/Indifferents, 30/Optimiser}
	\end{tikzpicture}
	\caption{Result of the cluster analysis: Four user groups}
	\label{fig:cluster} % \label has to be placed AFTER \caption to produce correct cross-references.
\end{figure}


\textbf{"Professionals" (48 \%):}
The Professionals are the biggest group. The members of this group are technically competent and interested in topics concerning energy.

The main characteristics are:
\begin{itemize}
	\item High proportion of persons having an abitur or university graduates
	\item Highest proportion of people in managerial positions, a quarter works (also) at home
	\item All household sizes (also households with children)
	\item Knowledge about energy
	\item High technical competence and interest in Technology. (Experience with home automation, a quarter has programming skills)
	\item Interested in sustainability
	\item Use of media or Internet is primarily for professional purposes
\end{itemize}

Typical segment behavior regarding home equipment:
 \begin{itemize}
 	\item "Reasonable" use of hot water ("I do not shower longer than necessary")
 	\item "Reasonable" use of lighting ("I turn down the light when I leave a room")
 	\item Make use of the "ECO-Button" (installed tool in the apartments of D12 which helps to save energy) when leaving the apartment
 \end{itemize}

Due to their technical expertise, their experience with home automation and their interest in energy issues they are the most appropriate target group for home automation and mobile application solutions. Rationally justified explanations and instructions for use meet their information style. Professionals also expect more detailed information in individual offers such as energy feedback.

\textbf{"Optimizer" (30 \%):}

The second largest segment comprises people who primarily aim to optimize energy costs. Optimizer have little knowledge about energy and are no technophiles.

The main characteristics are:
\begin{itemize}
	\item High proportion of persons having an abitur or university graduates
	\item Highest proportion of people in managerial positions
	\item More women
	\item All household sizes (also households with children)
	\item Interested in sustainability	
	\item Little to no knowledge about home automation
	\item no technophiles
	\item Use of media or Internet is not very noticeable
\end{itemize}

Typical segment behavior regarding home equipment:
\begin{itemize}
	\item Prefer to air manually rather than to make use of the automatic ventilation system
	\item A quarter never uses the "ECO-Button"
\end{itemize}

The use of the home furnishings indicates a poor understanding of their usability or less time of interaction with them. Due to their much lower competence in energy and technology compared to the professionals, the planned solutions and measures should focus very strongly on the following points:

\begin{itemize}
	\item Clear and concrete instructions for behavior, for example in the form of energy-saving tips or concrete, close to reality explanations and concrete benefits.
	\item Avoid technical language in communication and use personalized examples.
	\item Reduce energy feedback to essential information. Optimizer do not need detailed explanations. 
	\item Enable trouble shooting: Optimizer want a quick solution to an energy problem, as they do not want to spend lot of time on energy topics.
\end{itemize}

\textbf{"Indifferents" (13 \%):}

The "Indifferents" have low competence in energy and technology and no interest in energy topics or sustainability.

The main characteristics are:
\begin{itemize}
	\item Young segment
	\item High proportion of Non-workers
	\item No interest in sustainability
	\item Low technical competence (no experience with home automation)
	\item Information research and streaming is above average
\end{itemize}

Typical segment behavior regarding home equipment:
\begin{itemize}
	\item Hedonistic use of hot water: They enjoy taking long showers and baths
	\item Smallest number on different device types
	\item Little satisfaction with the provided air ventilation
\end{itemize}

The "Indifferents" have low interest in the research topic and it's solution in general. To address this group with the necessary knowledge and to awaken their interest for energy and sustainability, a bigger effort has to be done than for the above groups. A typical representative of this group is a person who has just moved out from the parental home and who now has to organize the household on his/her own and to develop independence.

\textbf{"Hedonists" (9 \%):}

The "Hedonists" are technical competent but are indifferent to energy and sustainability topics.

The main characteristics are:
\begin{itemize}
	\item Young segment
	\item More mens, more single households
	\item Technical competent and partly with programming skills
	\item Intensive use of mobile Apps and Internet
	\item Hedonistic use of gaming and social media
\end{itemize}

Typical segment behavior regarding home equipment:
\begin{itemize}
	\item Highest number on different device types
	\item Carefree use of lighting and hedonistic use of hot water
	\item Frequent use of "ECO-Button"
	\item High satisfaction with the provided air ventilation
	\item Weak identification with Seestadt
\end{itemize}

The youngest segment has good preconditions to make a good use of mobile application with feedback of their energy use. Nevertheless, the motivation to deal with energy topic is rather low. The hedonistic lifestyle with its strong convenience and comfort orientation is in the foreground. Despite the high usage of apps it may be difficult to win them around for energy feedback. The comfort gain is of great relevance.

\section{Usability Tests} mobile context
In order to avoid distorting of the research results the graphical user interface will be tested empirically with 4-5 usability tests, that means the usability is accessed by testing the interface with real users \cite{nielsen1994usability}.

\section{User study}
The design of the user study will follow the seminal guidelines for conducting case study research in software engineering as proposed by Runeson et al. \cite{runeson2012case}. The target group will consist of at least one user for each type of energy user. The study protocol will follow the check-lists for reading and reviewing case studies from H\"ost and Runeson \cite{host2007checklists}.

\section{Evaluation}
In this step the developed mobile app will be empirically evaluated against a valuation model in a user study to identify the success of the research. The evaluation model comprises of numerous Key Performance Indicators (KPIs). An extraction of these KPIs is listed in the following:
\begin{enumerate}
	\item More than 50 \% of all the users using the app state that the possibility of switching between different ways to display the information is useful
	\item More than 50 \% state, that they are more aware of what to do to avoid CO2 than before using the app
	\item More than 50 \% of the users state that they understand and get a feeling of how much CO2 they are producing
\end{enumerate}


Carbon dioxide awareness

\cite{mohammadmoradieffectiveness}

\section{Existing approaches}

\section{Comparison of existing approaches}
In this step, the market and competition analysis which was done when the problem arose will be done in more depth. The questions that shall be answered in this steps are the following.
\begin{itemize}
	\item Which applications are there within the topics of energy saving and CO2 awareness?
	\item Which approaches and visualizations do these applications make use of to increase awareness?
	\item How do these applications handle the users' level of education concerning energy units of measurements, such as kWh?
\end{itemize}

\section{Serious games}

The Energy Piggy Bank - A Serious Game for Energy Conservation

Serious games are games that educate, train, and inform

Serious games are gaining importance recently. These games aim at behavior change and education.

Hedin et al. \cite{Bjorn1165339} describe a serious game that shall help people learn more about their energy consumption. They designed the game according to the taxonomy of Bartles Player Types that constitute of four Types having different motivation for playing games.


They also evaluated the behaviour 

self-assessed future behaviour change 

The outcome of the work is a strong correlation between self-assessed future behavior change and perceived value/usevulness of the application independent of the player type.

Bartle Player Types

Serious games have attracted much attention recently and are used to in an engaging way support for example education and behavior change. In this paper, we present a serious game designed for helping people learn about their own energy consumption and to support behavior change towards more sustainable energy habits. We have designed the game for all the four Bartle Player Types, a taxonomy used to identify different motivations for playing games. Engagement of the participants has been evaluated using the Intrinsic Motivation Inventory, and we have measured self-assessed future behavior change. We found a statistically significant positive correlation between self-assessed future behavior change and perceived value/usefulness of the application independent of player type. Our study indicates the player type “Achievers” might perform better using this type of application and find it more enjoyable, but that it can be useful for learning energy conserving behavior independent of player type


\section{Persuasive System}
Tailoring and personalizing the content to the potential needs, interests, usage context or other factors is outlined by \cite{oinas2009persuasive} in the context of a Persuasive System. They studied how a persuasive system must be designed with tailored and personalized content to maximize the change in the user's behaviour. Although the outcome on the behaviour change is not relevant, the findings on the tailor aspects are highly interesting for the proposed thesis.


%The process of reconnecting is simply to stop. Stopping meditation 6 times a day for 10 seconds. Experiencing as a felt reality: Divinity
%Whatever you want to experience in your life: Be the source of experience for that for one another


 

% CHAPTER Methodology
\chapter{Methodology}

In this chapter the methodology including the used concepts such as survey for user segmentation, paper prototyping for usability testing and elicitation of requirements is described. Finally, we describe the design methods for the catalogue of design principles.

\section{Survey for User Segmentation}

The questionnaire that we created for finding test users for the Paper Prototyping Session follows the design guidelines of Andrews et al. \cite{andrews2007conducting}. The guidelines say that electronic surveys should be designed to...

\begin{itemize}
	\item support multiple browsers and platforms \cite{yun2000comparative}
	\item prevent submitting multiple times \cite{yun2000comparative}
	\item to present questions in an adaptive or logical manner \cite{kehoe1997eighth}
	\item allow saving the work in long questionnaires with more than 50 questions \cite{smith1997casting}
	\item collect both quantified selection option answers and narrative type question answers \cite{yun2000comparative}
	\item have the possibility to thank the users for completing the survey \cite{smith1997casting}.
\end{itemize}

Google Form is a web application out of the Google Web Apps that follows all these guidelines, which was the reason for choosing it for our survey.

As the motivation to find subjects who completes a survey increases as the question difficulty increases \cite{andrews2007conducting}, when the aim is to have numerous replied questionnaires the survey should comprise of simple and not to much questions.

Generally online survey platforms offer convenient and reliable data management \cite{carbonaro2000design}. By design, Google Form protects against the loss of data and facilitates data transfer into a database, in this case Google Spreadsheets for analysis.

Before sending out the survey and after deciding on the survey tools, contents and platforms it is very important to carry out a pilot \cite{lumsden2007online}.

\subsection{Evaluation of the Questionnaires}

The user segments on which this thesis builds upon are described in \ref{chap:usersegmentation}. Smart Cities Demo Aspern did a survey and used cluster analysis to define clusters. These clusters had distinct features. The answer of one user of the questionnaire can then be evaluated against each user segment with it's distinct features. This evaluation amounts to a correspondence of one answer set to a user segment.

According to Kazi and Khalid \cite{kazi2012questionnaire} there are three types of validity, which is the degree to which an assessment measures what it is supposed to measure. The three types are content validity, criterion-related validity and construct validity. The validation technique for identifying the correspondence of a user to a user segment is the criterion-related validity as it best describes the equivalence to the segment characteristics.

\section{Elicitation of Requirements with Paper Prototyping}
  For the Paper Prototype a Step-by-Step guide proposed by Arnowitz et al. \cite{arnowitz2010effective} is used. To create a Paper Prototype the following steps should be done:

\begin{enumerate}
	\item \textbf{Create scenario}. Before starting to draw anything the main user goals and tasks have to be portrayed. This can be done in a scenario narration.
	
	\item \textbf{Inventory UI elements}. The next step is to make a checklist of all UI elements that may be needed to support the scenario
	.
	\item \textbf{Create UI elements}. All the UI elements from the checklist from the previous step are now created in paper form. There are a lot of tools and materials that can come in handy at this step. The following list of materials might help the process: paper, sticky notes, whiteboard, sketchbook, notebook, napkin, cards, overhead sheet, cardboard, carton, scissors, markers, UI stencil, correction fluid and tape and transparency sheet. 
	
	\item \textbf{Run through scenario}. In this step a dry-run through the scenario with the paper prototype should be done and missing parts should be found an recreated.
	
	\item \textbf{Internal review}. The last step in the first round is the internal review with the team where the audience is defined, the goals for each version of the prototype is reviewed, the expectation of the reviewers are found out and the next steps are planned.
\end{enumerate}

The next Step-By-Step Guide is following the first. It was also proposed by Arnowitz et al. \cite{arnowitz2010effective} and is for testing the Paper Prototype:

\begin{enumerate}
	\item \textbf{Revise scenario}. The internal review may have uncovered some tweaks that you want to change. Be careful for changes at the scenario as it may cause a ripple effect which can lead to necessary changes in user interface elements or even new screens. Keeping changes to minimum is recommended. If changes are necessary keep in mind that this can lead to non comparable results in the end.
	
	
	\item \textbf{Revise inventory UI elements}.
	Until now maybe multiple run throughs through the prototype have been done and you noticed that some vital pieces of the interface are missing. Now is a good time to check completeness of the UI elements checklist. Developing a set of UI elements for cases that you did not anticipate may be also useful.
	
	\item \textbf{Create UI elements}.
	Check if the collection of the UI elements is still complete and create some more if needed.
	
	\item \textbf{Pilot run through scenario}. Before presenting the prototype to the user it has to be tried out first. You can give the Prototype to anyone, e.g. a team member, to try it out. The aim here is to find missing pieces to be prepared for everything they do. The run through will ensure that you haven't created a half-baked prototype.
	
	\item \textbf{Internal review}. 	
	In this step the scenario and the prototype supplies are revised again with the team. Also the goals and the expectation of the reviewers are revised.
	
	\item \textbf{Prepare Kit}. Before running the prototype session the papers have to be arranged in a way that makes it easy to find the various UI elements. Also blank paper, sticky notes and pens should be prepared for further ideas.
	
	\item \textbf{The Prototype Session}. The user study session is an interactive process where one ore more participants and a facilitator are involved. In a dialogue the participant completes tasks provided by the facilitator. The session is used to get user opinions about early design and task flow ideas represented on paper. The sessions are typically recorded for later examination. The feedback from the users show what they expect from the app which is of great value for the implementation later on \cite{snyder2003paper}. Weiss  \cite[p.~144]{weiss2003handheld} proposes to invite not only one, but two respondents at a time for paper prototype usability tests. He mentions that two respondents feel more comfortable in the casual environment that paper prototyping creates, whereas one single respondent can easily become overwhelmed by the experience.
	
	\item \textbf{Reiterate}. After each prototype session an review and evaluation about what went good and what bad can be done. Although it might be tempting to change things after each session, it is better to wait until all the planned user sessions are done to do an overall comparative review at the end.
\end{enumerate}

\section{Creating a catalogue of design guidelines}

A common challenge is to interpret the results of empirical studies and derive design guidelines which are not too specific but also not too general to make them applicable without additional interpretation effort. The methodology for the deduction of design guidelines for this thesis is inspired by De Bruijn's and Spence's framework for theory-based interaction design. 

\begin{figure}[h]
	\centering
	\includegraphics[width=\textwidth]{design_action}
	\caption{Exemplary design action of De Bruijn and Spence \cite{de2008new}}
	\label{fig:designaction} % \label has to be placed AFTER \caption (or \subcaption) to produce correct cross-references.
\end{figure}

The design action is headed by an \textbf{identifier} and a \textbf{title} indicating as clearly as possible the expected result of applying the design action. The \textbf{description} clarifies the brief title, followed by the \textbf{effect} that the design action will have. The design action further includes an \textbf{upside} and \textbf{downside} section that describes advantage and trade-offs respectively. The \textbf{Issues} sections considers issues that are neither positive nor negative. In the last part of the design action the \textbf{Theory} is provided as an opportunity for the designer to dive deeper. Nevertheless, the designer does not necessarily need to understand it in order to apply the guideline.



% CHAPTER Implementation
\chapter{Implementation}

\section{Questionnaire for user segmentation}

To find test users for each of the four user segments as explained in \ref{chap:usersegmentation} we conducted a user survey. The user survey leaned on to the first questionnaire of the quantitative study of Aspern Smart City Research. The original questionnaire of ASCR comprised of 48 questions. The factor analysis of the returned questionnaire identified the four dimensions: Comfort-centered, Technology-centered, Data Sensibility and Living in Seestadt. The following cluster analysis found out four segments. For the definition of the segments only two of the four factors were really relevant for describing the characteristics of a user group. For our study we focused on these two factors which are the comfort and the technology orientation. So, we took all the questions of the original questionnaire which answers were identified by the cluster analysis to be significant for the user segmentation. From the 48 questions of the original questionnaire only ten were relevant for allocating a user to a user segment.

FRAGEBOGEN AN DIESER STELLE EINFÜGEN ODER FRAGEBOGEN IN DEN APPENDIX???


For creating the survey and sending it to different people we used Google Forms. We sent the questionnaire to 57 people, trying to have a good distribution of different ages, educational levels, jobs and interests. 31 questionnaires were returned. Answering the 10 questions only took about one minute.

For evaluating the response we used Google Spreadsheet and Microsoft Excel. The answers of each person was evaluated against the characteristics of the four user segments. Of course not every user could easily be assigned to exactly one user segment. For each user the correspondence to each of the four user segments was calculated and expressed in percent. The ones who had a clear correspondence of more than 50 \% to one user type were chosen as test subjects for the paper prototyping, the usability tests and the user study later on. So at least one user for each user type was chosen. The next step was the Paper Prototype.


\section{Paper Prototyping}
There, hand-sketched drafts will be drawn, showing the app with menus, dialogue boxes, notifications and buttons. Then, different tasks that can be done with the app shall be defined. These tasks are then conducted by the users.

As described before in \ref{section:paperprototyping} we follow the Step-by-Step guide of Arnowitz et al. \cite{arnowitz2010effective} to create a Paper Prototype:


\begin{enumerate}
	\item \textbf{Create scenario}.
	
	The main goal of the mobile application that shall be developed is to give feedback about how much electricity, water and heating a user consumed, how much carbon-dioxide was produced and how the values can be made better. The screens shall be adapted to the user type to make the information and tips more attractive to a user's interests.
	
	\subitem{\textbf{Professionals}}: The user study revealed that Professionals have high interest in energy issues. As they are deep into the topic and prefer more detailed information in individual offers they can have a look on the energy consumption on a very detailed level, such as a consumption rate on a granularity of minutes.
 
	\begin{itemize}
		\item Task: Have a look on your consumption rate of the last week/month/year
	\end{itemize}
	
	Professionals should have a possibility within the application to compare energy consumptions of different time intervals.
	
	\begin{itemize}
		\item Task: Compare your consumption rate of the last week with the consumption rate of the same week one year earlier.
	\end{itemize}

	Professionals also like to compare themselves to others. Comparing his or her average energy consumption to others should also be possible.
	
	\begin{itemize}
		\item Task: Compare your consumption rate of the last week with the average consumption rate of your neighbours
	\end{itemize}
	
	Professionals also like rationally justified explanations and instructions for use. Notifications on a daily basis give tips on saving energy or CO2, give concrete instructions for use and provides deeper information in Energy topics.
	
	\begin{itemize}
		\item Task: Find tips on how to save more energy
	\end{itemize}
	
	\subitem{\textbf{Professionals}}: Optimizer primarily aim at optimizing energy costs, so the app should give easy to find tips on how to save energy and therefore costs. Optimizers prefer less time of interaction. As Optimizer rather like unclear instructions, the notifications on a daily basis also should give concrete instructions on how to save energy or CO2.
	
	\begin{itemize}
		\item Task: Find out what to do to save costs for electricity
	\end{itemize}
	
	 Professionals like to know the concrete benefits of a certain behavior change. The explanations shall be as close to reality as possible and technical language shall be avoided. The energy feedback is reduced to essential information and the detailed graphs for energy consumption that the Professionals get are not visible at a first glance for an Optimizer. The saved costs after a behaviour change shall also be visible to provide some kind of reward for the new habits.
	 
	 \begin{itemize}
	 	\item Task: Find out how much you have saved in the last week
	 \end{itemize}
	
	For Optimizer also trouble shooting shall be easily accessible, in order to reduce the time they are spending with the application and not to loose them on the way.
	
	\begin{itemize}
		\item Task: Report a problem
	\end{itemize}
	
	\subitem{\textbf{Indifferents}}:
	The Indifferents have low interest in energy topics in general, so the main requirement of the application for this type of user is in the first run to sensitize them for the topic, to raise awareness and to make electricity and CO2 saving appealing to them. 
	
	To awaken their interest for energy and sustainability a gamification approach will be used. For opening the application once a day the user earns points. Points are also earned for clicking on notifications and reading the article. Tips for saving energy or CO2 should not concern longer usage of laptops or entertainment screens, as streaming and use of social media is an important leisure activity for Indifferents.
	
	\begin{itemize}
		\item Task: Earn points by interacting with the App
	\end{itemize}
	
	\textit{Hedonists}:
	The youngest segment, the Hedonists, are keen on developing technical solutions. The interest in technology can be used to give instructions for programming technical devices and using home automation. The primary motive for the Hedonists is not to save energy but the interest in technology. This will be considered in the notifications and tips of the day. The hedonistic lifestyle with its strong convenience and comfort orientation is in the foreground.
	
	For a hedonist the comfort gain is of great relevance. Programming and establishing home automation aspects is a great interface between the aim of saving energy and the affinity of technology.
	
	
	\item \textbf{Inventory UI elements}. The next step is to make a checklist of all UI elements that may be needed to support the scenario.
	\item \textbf{Create UI elements}. All the UI elements from the checklist from the previous step are now created in paper form. There are a lot of tools and materials that can come in handy at this step. The following list of materials might help the process: paper, sticky notes,
	whiteboard, sketchbook, notebook, napkin, cards, overhead sheet, cardboard, carton, scissors, markers, UI stencil, correction fluid and tape and transparency sheet. 
	\item \textbf{Run through scenario}. In this step a dry-run through the scenario with the paper prototype should be done and missing parts should be found an recreated.
	\item \textbf{Internal review}. The last step in the first round is the internal review with the team where the audience is defined, the goals for each version of the prototype is reviewed, the expectation of the reviewers are found out and the next steps are planned.
\end{enumerate}

The next Step-By-Step Guide is following the first. It was also proposed by Arnowitz et al. \cite{arnowitz2010effective} and is for testing the Paper Prototype:

\begin{enumerate}
	\item \textbf{Revise scenario}. The internal review may have uncovered some tweaks that you want to change. Be careful for changes at the scenario as it may cause a ripple effect which can lead to necessary changes in user interface elements or even new screens. Keeping changes to minimum is recommended. If changes are necessary keep in mind that this may result in the inventory of user interface.
	\item \textbf{Revise inventory UI elements}.
	Until now maybe multiple run throughs through the prototype have been done and you noticed that some vital pieces of the interface are missing. Now is a good time to check completeness of the UI elements checklist. Developing a set of UI elements for cases that you did not anticipate may be also useful.
	\item \textbf{Create UI elements}.
	
	
	\item \textbf{Pilot run through scenario}. Before presenting the prototype to the user it has to be tried out first. You can give the Prototype to anyone, e.g. a team member, to try it out. The aim here is to find missing pieces to be prepared for everything they do. The run through will ensure that you haven't created a half-baked prototype.
	
	\item \textbf{Internal review}. 
	
	The last step in the first round is the internal review with the team where the audience is defined, the goals for each version of the prototype is reviewed, the expectation of the reviewers are found out and the next steps are planned.
	
	
	\item \textbf{Prepare Kit}. Before running the prototype session the papers have to be arranged in a way that makes it easy to find the various UI elements. Also blank paper, sticky notes and pens should be prepared for further ideas.
	\item \textbf{The Prototype Session}. The user study session is an interactive process where one ore more participants and a facilitator are involved. In a dialogue the participant completes tasks provided by the facilitator. The session is used to get user opinions about early design and task flow ideas represented on paper. The sessions are typically recorded for later examination. The feedback from the users show what they expect from the app which is of great value for the implementation later on \cite{snyder2003paper}. Weiss  \cite[p.~144]{weiss2003handheld} proposes to invite not only one, but two respondents at a time for paper prototype usability tests. He mentions that two respondents feel more comfortable in the casual environment that paper prototyping creates, whereas one single respondent can easily become overwhelmed by the experience.
	\item \textbf{Reiterate}. After each prototype session an review and evaluation about what went good and what bad can be done. Although it might be tempting to change things after each session, it is better to wait until all the planned user sessions are done to do an overall comparative review at the end.
\end{enumerate}


% CHAPTER Evaluation
\input{evaluation.tex}

% CHAPTER Critical reflection
\chapter{Critical reflection}

Developing applications that give eco-feedback and are persuasive for behaviour change is not a straightforward but complex task. Different aspects have to be considered, such as usability issues, mobile context restrictions, adaptive interfaces, persuasive technology, environmental psychology, human computer interaction and preferences and characteristics of users.

\section{Comparison with related work}

% CHAPTER Summary
\chapter{Summary}

A questionnaire for the determination of a user type, Paper prototypes for a mobile application that adapts to a user type and design guidelines for such a mobile application are the contribution of this thesis. Furthermore, the 

\section{Future work}

% CHAPTER Literature
% \input{literature_summary.tex}

\backmatter

% Use an optional list of figures.
\listoffigures % Starred version, i.e., \listoffigures*, removes the toc entry.

% Use an optional list of tables.
\cleardoublepage % Start list of tables on the next empty right hand page.
\listoftables % Starred version, i.e., \listoftables*, removes the toc entry.

% Use an optional list of alogrithms.
\listofalgorithms
\addcontentsline{toc}{chapter}{List of Algorithms}

% Add an index.
\printindex

% Add a glossary.
\printglossaries

% Add a bibliography.
\bibliographystyle{alpha}
\bibliography{intro}

\end{document}