%---------------------------------------------------------------------------
\chapter{State of the Art}
%---------------------------------------------------------------------------

In the following sections the theoretical background for the topics that this thesis deals with will be presented. In particular, usability engineering especially different forms of visualizations and graphical user interfaces in the scope of mobile applications. That implies a research about paper prototyping, usability testing in the mobile context as well as user classification for the definition of user groups and carbon dioxide awareness in general. Finally, we will have a look on existing approaches, such as serious games and a comparative analysis of alternatives.

\section{Usability engineering}

The International Organization for Standardization (ISO) defines usability as the "Extent to which a product can be used by specified users to achieve specified goals with effectiveness, efficiency and satisfaction in a specified context of use" \cite{bevan1998iso}. This definition comprises three measurable attributes which are the following:

\begin{itemize}
	\item \textbf{Effectiveness}: Accuracy and completeness with which users achieve specified goals.
	\item \textbf{Efficiency}: Resources expended in relation to the accuracy and completeness with which users achieve goals.	
	\item \textbf{Satisfaction}: Freedom from discomfort, and positive attitudes towards the use of the product.
\end{itemize}

The ISO standard also identifies three factors that should be considered when evaluating usability:
\begin{itemize}
	\item \textbf{User}: Person who interacts with the product.
	\item \textbf{Goal}: Intended outcome.
	\item \textbf{Context of use}: Users, tasks, equipment (hardware, software and materials), and the physical and social environments in which a product is used.
\end{itemize}

Nielsen \cite{nielsen1994usability} also identified five attributes of usability and factors having an impact on how the user interacts with a system. In addition to the above ones Nielsen \cite{nielsen1994} states:

\begin{itemize}
	
	\item \textbf{Learnability}: The user should get work done rapidly which is possible if the system is easy to use.
	\item \textbf{Efficiency}: Once the user has learned to operate with the system, the productivity should be high.
	\item \textbf{Memorability}: In case a user does not use the system in a longer period, it should, nevertheless, be easy remembered without having to learn everything all over again.	
	\item \textbf{Errors}: When using the system, the user makes few errors and is able to return and recover easily after an error. Further, catastrophic errors must not occur.
	\item \textbf{Satisfaction}: The system is highly accepted as the user has positive attitudes towards the system and finds it pleasant to use.
\end{itemize}


\subsection{Usability engineering in the context of mobile applications}

The focus on usability and interaction between human and hand-held electronic devices has its origin within the emergence of mobile devices. The approach of Nielsen, mentioned above, was expanded with the scope of mobile applications by Zhang and Adipat \cite{zhang2005challenges} who highlighted a number of issues by the advent of mobile devices. The issues mentioned are:
\begin{itemize}
	
	\item mobile context
	\item connectivity
	\item small screen
	\item different display resolution
	\item limited processing capability and power and
	\item data entry methods
	
\end{itemize}

Zhang et al. mention that these restrictions are especially a problem when it comes to usability testing methods, as all these issues must be considered in order to select an appropriate research methodology and it must be kept in mind that contextual factors on perceived usability can occur when they are not considered in a study.

Harisson et al. \cite{harrison2013usability} build up on the terms mentioned before and introduced a PACMAD (People At the Centre of Mobile Application Development) model which was designed to address the limitations of existing usability models when applied to mobile devices. PACMAD extends the theories of usability with more aspects such as \textit{user task} and \textit{context of use}. The existing usability models such as those proposed by Nielsen \cite{nielsen1994usability} and ISO \cite{bevan1998iso} also recognise these factors as crucial parts on which the successfulness of the usability of an application depends. The difference is that PACMAD includes all the factors into one model to ensure a complete usability evaluation.

Deka \cite{deka2016data} discusses how data-driven approaches are tools for mobile app design. A relevant field mentioned is interaction mining, that captures the static part, such as layouts and visual details, as well as the dynamic part, like user flows and motion details, of app design.

The decades of research in adaptive user interfaces were summarized by Gajos et al. \cite{gajos2008decision}. They conclude that personalized user interfaces have the ability to improve user satisfaction and performance, when the interface is adapted to the device, task, preferences and abilities of the person. To automatically generate user interfaces they use decision-theoretic optimization including functional specifications of the interface, constraints of the devices e.g. screen size and a list of available interactors, a typical usage trace and a cost function. The cost function holds user preferences and expected speed of operation.


Decision-Theoretic User Interface Generation

\section{Elicitation of requirements with Paper Prototyping}
The second step is to do "Paper Prototyping" in order to elicit the requirements for the graphical user interfaces and overall for the CO2 awareness app. According to \cite{lancaster2004paper} the numerous benefits of early usability studies are vastly superior. It may seem low-tech, but conducting usability tests at this step show what users really expect on a quite detailed level which gives maximum feedback for minimum effort \cite{weiss2003handheld}.

At first a group of people containing at least one user for each user type will be put together.  Next, hand-sketched drafts will be drawn, showing the app with menus, dialog boxes, notifications and buttons. Then, different tasks that can be done with the app shall be defined. These tasks are then conducted by the users. The feedback from the users show what they expect from the app which is of great value for the implementation later on \cite{snyder2003paper}.




\section{User Classification}


Weiss \cite{weiss2003handheld} discussed the balance of ease of use compared with the ease of learning. A huge emphasis is laid on the first, and according to Weiss, the most important step in the design and development process, the understanding of the audience. The purpose of the audience definition is to describe the target group, its' traits and ranges.





\section{Usability Tests} mobile context
In order to avoid distorting of the research results the graphical user interface will be tested empirically with 4-5 usability tests, that means the usability is accessed by testing the interface with real users \cite{nielsen1994usability}.

\section{User study}
The design of the user study will follow the seminal guidelines for conducting case study research in software engineering as proposed by Runeson et al. \cite{runeson2012case}. The target group will consist of at least one user for each type of energy user. The study protocol will follow the check-lists for reading and reviewing case studies from H\"ost and Runeson \cite{host2007checklists}.

\section{Evaluation}
In this step the developed mobile app will be empirically evaluated against a valuation model in a user study to identify the success of the research. The evaluation model comprises of numerous Key Performance Indicators (KPIs). An extraction of these KPIs is listed in the following:
\begin{enumerate}
	\item More than 50 \% of all the users using the app state that the possibility of switching between different ways to display the information is useful
	\item More than 50 \% state, that they are more aware of what to do to avoid CO2 than before using the app
	\item More than 50 \% of the users state that they understand and get a feeling of how much CO2 they are producing
\end{enumerate}


Carbon dioxide awareness

\cite{mohammadmoradieffectiveness}

\section{Existing approaches}

\section{Comparative analysis of alternatives and comparison of existing approaches}
In this step, the market and competition analysis which was done when the problem arose will be done in more depth. The questions that shall be answered in this steps are the following.
\begin{itemize}
	\item Which applications are there within the topics of energy saving and CO2 awareness?
	\item Which approaches and visualizations do these applications make use of to increase awareness?
	\item How do these applications handle the users' level of education concerning energy units of measurements, such as kWh?
\end{itemize}

\section{Serious games}

The Energy Piggy Bank - A Serious Game for Energy Conservation

Serious games are games that educate, train, and inform

Serious games are gaining importance recently. These games aim at behavior change and education.

Hedin et al. \cite{Bjorn1165339} describe a serious game that shall help people learn more about their energy consumption. They designed the game according to the taxonomy of Bartles Player Types that constitute of four Types having different motivation for playing games.


They also evaluated the behaviour 

self-assessed future behaviour change 

The outcome of the work is a strong correlation between self-assessed future behavior change and perceived value/usevulness of the application independent of the player type.

Bartle Player Types

Serious games have attracted much attention recently and are used to in an engaging way support for example education and behavior change. In this paper, we present a serious game designed for helping people learn about their own energy consumption and to support behavior change towards more sustainable energy habits. We have designed the game for all the four Bartle Player Types, a taxonomy used to identify different motivations for playing games. Engagement of the participants has been evaluated using the Intrinsic Motivation Inventory, and we have measured self-assessed future behavior change. We found a statistically significant positive correlation between self-assessed future behavior change and perceived value/usefulness of the application independent of player type. Our study indicates the player type “Achievers” might perform better using this type of application and find it more enjoyable, but that it can be useful for learning energy conserving behavior independent of player type


\section{Persuasive System}
Tailoring and personalizing the content to the potential needs, interests, usage context or other factors is outlined by \cite{oinas2009persuasive} in the context of a Persuasive System. They studied how a persuasive system must be designed with tailored and personalized content to maximize the change in the user's behaviour. Although the outcome on the behaviour change is not relevant, the findings on the tailor aspects are highly interesting for the proposed thesis.

\newpage
x
\newpage
x



%The process of reconnecting is simply to stop. Stopping meditation 6 times a day for 10 seconds. Experiencing as a felt reality: Divinity

%Whatever you want to experience in your life: Be the source of experience for that for one another

Eukalyptus und Lavendel
\newpage
Lemongras and Teatree
\newpage
x
\newpage

