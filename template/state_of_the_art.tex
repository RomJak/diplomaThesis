%---------------------------------------------------------------------------
\chapter{State of the Art}
%---------------------------------------------------------------------------

In the following sections the theoretical background for the topics that this thesis deals with will be presented. In particular, usability engineering especially different forms of visualizations and graphical user interfaces in the scope of mobile applications. That implies a research about paper prototyping, usability testing in the mobile context as well as user classification for the definition of user groups and carbon dioxide awareness in general. Finally, we will have a look on existing approaches, such as serious games and a comparative analysis of alternatives.

\section{Usability engineering}

The International Organization for Standardization (ISO) defines usability as the "Extent to which a product can be used by specified users to achieve specified goals with effectiveness, efficiency and satisfaction in a specified context of use" \cite{bevan1998iso}. This definition comprises three measurable attributes which are the following:

\begin{itemize}
	\item \textbf{Effectiveness}: Accuracy and completeness with which users achieve specified goals.
	\item \textbf{Efficiency}: Resources expended in relation to the accuracy and completeness with which users achieve goals.	
	\item \textbf{Satisfaction}: Freedom from discomfort, and positive attitudes towards the use of the product.
\end{itemize}

The ISO standard also identifies three factors that should be considered when evaluating usability:
\begin{itemize}
	\item \textbf{User}: Person who interacts with the product.
	\item \textbf{Goal}: Intended outcome.
	\item \textbf{Context of use}: Users, tasks, equipment (hardware, software and materials), and the physical and social environments in which a product is used.
\end{itemize}

Nielsen \cite{nielsen1994usability} also identified five attributes of usability and factors having an impact on how the user interacts with a system. In addition to the above ones Nielsen \cite{nielsen1994} states:

\begin{itemize}
	
	\item \textbf{Learnability}: The user should get work done rapidly which is possible if the system is easy to use.
	\item \textbf{Efficiency}: Once the user has learned to operate with the system, the productivity should be high.
	\item \textbf{Memorability}: In case a user does not use the system in a longer period, it should, nevertheless, be easy remembered without having to learn everything all over again.	
	\item \textbf{Errors}: When using the system, the user makes few errors and is able to return and recover easily after an error. Further, catastrophic errors must not occur.
	\item \textbf{Satisfaction}: The system is highly accepted as the user has positive attitudes towards the system and finds it pleasant to use.
\end{itemize}


\subsection{Usability engineering in the context of mobile applications}

The focus on usability and interaction between human and hand-held electronic devices has its origin within the emergence of mobile devices. The approach of Nielsen, mentioned above, was expanded with the scope of mobile applications by Zhang and Adipat \cite{zhang2005challenges} who highlighted a number of issues by the advent of mobile devices. The issues mentioned are:
\begin{itemize}
	
	\item mobile context
	\item connectivity
	\item small screen
	\item different display resolution
	\item limited processing capability and power and
	\item data entry methods
	
\end{itemize}

Zhang et al. mention that these restrictions are especially a problem when it comes to usability testing methods, as all these issues must be considered in order to select an appropriate research methodology. It must be kept in mind that contextual factors on perceived usability can occur when they are not considered in a study.

Harisson et al. \cite{harrison2013usability} build up on the terms mentioned before and introduced a PACMAD (People At the Centre of Mobile Application Development) model which was designed to address the limitations of existing usability models when applied to mobile devices. PACMAD extends the theories of usability with more aspects such as \textit{user task} and \textit{context of use}. The existing usability models such as those proposed by Nielsen \cite{nielsen1994usability} and ISO \cite{bevan1998iso} also recognize these factors as crucial parts on which the successfulness of the usability of an application depends. The difference is that PACMAD includes all the factors into one model to ensure a complete usability evaluation.

Deka \cite{deka2016data} discusses how data-driven approaches are tools for mobile app design. A relevant field mentioned is interaction mining, that captures the static part, such as layouts and visual details, as well as the dynamic part, like user flows and motion details, of app design.

Fogarty and Hudson \cite{fogarty2003gadget} presented an experimental toolkit to support optimization for interface and display generation. This approach


The decades of research in adaptive user interfaces were summarized by Gajos et al. \cite{gajos2008decision}. They conclude that personalized user interfaces have the ability to improve user satisfaction and performance, when the interface is adapted to the device, task, preferences and abilities of a person. To automatically generate user interfaces they use decision-theoretic optimization which includes functional specifications of the interface, constraints of the devices e.g. screen size and a list of available interactors, a typical usage trace and a cost function. The cost function holds user preferences and the expected speed of operation.

\section{Elicitation of requirements with Paper Prototyping}
In order to elicit the requirements for the graphical user interfaces and overall for the CO2 awareness app we make use of Paper Prototyping. According to \cite{lancaster2004paper} the numerous benefits of early usability studies are vastly superior. Besides saving time and money by solving problems before the implementation even begins, Paper Prototyping stimulates creativity as it allows to experience with different ideas before committing to one \cite{snyder2003paper}. It may seem low-tech, but conducting usability tests at this step show what users really expect on a quite detailed level which gives maximum feedback for minimum effort \cite{weiss2003handheld}. 

A highly recommended introduction into effective prototyping is provided by Arnowitz et al. \cite{arnowitz2010effective} as well as by Bernard and Summers \cite{bernard2010dynamic} which inducted into dynamic prototyping. Arnowitz et al. \cite{arnowitz2010effective} proposed the following Step-by-Step guide to create a Paper Prototype:

\begin{enumerate}
	\item \textbf{Create scenario}. Before starting to draw anything the main user goals and tasks have to be portrayed. This can be done in a scenario narration.
	\item \textbf{Inventory UI elements}. The next step is to make a checklist of all UI elements that may be needed to support the scenario.
	\item \textbf{Create UI elements}. All the UI elements from the checklist from the previous step shall be created in paper form.
	\item \textbf{Run through scenario}. In this step a dry-run through the scenario with the paper prototype should be done and missing parts should be found an recreated.
	\item \textbf{Internal review}. The last step in the first round is the internal review with the team where the audience is defined, the goals for each version of the prototype is reviewed, the expectation of the reviewers are found out and the next steps are planned.
\end{enumerate}

The next Step-By-Step Guide is following the first. It was also proposed by Arnowitz et al. \cite{arnowitz2010effective} and is for testing the Paper Prototype.

\begin{enumerate}
	\item \textbf{Revise scenario}. The internal review may have uncovered some tweaks that you want to change. Be careful for changes at the scenario as it may cause a ripple effect which can lead to necessary changes in user interface elements or even new screens. Keeping changes to minimum is recommended. If changes are necessary keep in mind that this may result in the inventory of user interface.
	\item \textbf{Revise inventory UI elements}.
	Until now maybe multiple run throughs through the prototype have been done and you noticed that some vital pieces of the interface are missing. Now is a good time to check completeness of the UI elements checklist. Developing a set of UI elements for cases that you did not anticipate may be also useful.
	\item \textbf{Create UI elements}.
	
	
	\item \textbf{Pilot run through scenario}. Before presenting the prototype to the user it has to be tried out first. You can give the Prototype to anyone, e.g. a team member, to try it out. The aim here is to find missing pieces to be prepared for everything they do. The run through will ensure that you haven't created a half-baked prototype.
	
	\item \textbf{Internal review}. 
	
	The last step in the first round is the internal review with the team where the audience is defined, the goals for each version of the prototype is reviewed, the expectation of the reviewers are found out and the next steps are planned.
	
	
	\item \textbf{Prepare Kit}. Before running the prototype session the papers have to be arranged in a way that makes it easy to find the various UI elements. Also blank paper, sticky notes and pens should be prepared for further ideas.
	\item \textbf{The Prototype Session}. The user study session is an interactive process where one ore more participants and a facilitator are involved. In a dialogue the participant completes tasks provided by the facilitator. The session is used to get user opinions about early design and task flow ideas represented on paper. The sessions are typically recorded for later examination. The feedback from the users show what they expect from the app which is of great value for the implementation later on \cite{snyder2003paper}. Weiss  \cite[p.~144]{weiss2003handheld} proposes to invite not only one, but two respondents at a time for paper prototype usability tests. He mentions that two respondents feel more comfortable in the casual environment that paper prototyping creates, whereas one single respondent can easily become overwhelmed by the experience.
	\item \textbf{Reiterate}. After each prototype session an review and evaluation about what went good and what bad can be done. Although it might be tempting to change things after each session, it is better to wait until all the planned user sessions are done to do an overall comparative review at the end.
\end{enumerate}






\section{User Classification}


Weiss \cite{weiss2003handheld} discussed the balance of ease of use compared to the ease of learning. A huge emphasis is laid on the first, and according to Weiss, the most important step in the design and development process, the understanding of the audience. The purpose of the audience definition is to describe the target group, its' traits and ranges.

\subsection{User Segmentation according to Smart Cities Demo Aspern}

Aspern Smart City Research GmbH \& Co KG (ASCR) also lays emphasis on understanding the user. The research group defines a smart user as a person who has the knowledge for sustainable decisions in relation to his or her lifestyle. Saving CO2 and energy should be the overall goals of a smart user.

Nevertheless, not all smart users are the same and not all share the same state of knowledge or interest. Therefore, in 2015 ASCR conducted a socio-scientific study to find out how much know-how a smart user has in the field of technology and energy and also how much interest they have in the topics of energy and sustainability. The research was done in an apartment block named D12, where the possibility to test solutions is given, as the apartments in this block are equipped with systems that collect data including

\begin{itemize}
	\item electricity consumption
	\item room temperature
	\item warm/cold water consumption and
	\item air quality.
\end{itemize}

Over half of the households in the apartment block D12 agreed on making their data available for research purposes and to participate in surveys and workshops. In total, 85 households took part in the study in 2015. In the starting phase two studies were done. At first a qualitative study with personal interviews with selected tenants of the building D12 was done followed by a quantitative study with written questionnaires. One outcome of these studies was the segmentation of users into groups. Different types of users were clustered into four segments according to their state of knowledge and their interest in technology and energy. The user groups also serve as target groups for the development of new technology solutions such as home automation, mobile application and for the development of range of services. The segmentation into groups also makes communication easier as the used methods of communicaiton can be tailored to the needs of a group.

The qualitative study with its interviews was done before the tenants moved into the apartments in Seestadt. Surprisingly the majority stated that it has basic knowledge for the interpretation of the energy consumption and energy data in general. Often they stated that they do not know how much on kilowatt hour is. In most cases the main source of information for energy topics is the energy consumption calculation. Unfortunately the calculation does not state the behaviour or the devices which use up the most energy. Exactly these two aspects are the most wished information for the users when it comes to saving energy.

The aim of a segmentation in its statistical way is to find distinct groups with significant differences \cite{punj1983cluster}. Within a group the characteristics should be homogeneous. An established way for segmentation in statistics is to do two statistic procedures, beginning with a factor analysis, followed by a cluster analysis \cite{tuffery2011data}.

The factor analysis reduces dimensions \cite{williams2010exploratory}. In the quantitative questionnaires multiple variables are collected and in the factor analysis these variables are reduced to so called latent variables or factors. Therefore, the factor analysis shows which dimensions are underlying the whole questionnaire.

In the socio economic study of ASCR an explorative non-rotating factor analysis was calculated. Afterwards the scree test showed the amount of factors, which was in this case four. In terms of content the analysis of the factor showed the following dimensions:

\begin{itemize}
	\item \textbf{Comfort-centered}: This factor covers aspects like home automation, energy relevant user behavior such as lighting and circulation behavior and hot water usage. 
	\item \textbf{Technology-centered}: Also covers aspects like home automation but more with the sense of interest in the technology rather than the comfort aspect.
	\item \textbf{Data sensibility}: Concerns regarding the further use of the collected data.
	\item \textbf{Living in Seestadt}: The aspect of living in the Seestadt as an extra dimension shows that it is some kind of prestige to live there.
\end{itemize}

Finally a cluster analysis was done to identify the user segments. Cluster analysis is an exploratory process with the aim of finding groups of similar objects \cite{tuffery2011data}. Different hierarchical analysis were calculated to find an appropriate amount of clusters. Appropriate means in this case having an big enough group of cases/persons and groups having distinct features. The data set comprised 121 handed back questionnaires and the cluster analysis could identify four clusters. The four clusters correspond to the four user groups. The result of the cluster analysis is shown in ~\ref{fig:cluster} and explained in the following.

\begin{figure}[h]
	\centering
	\begin{tikzpicture}
	\pie [rotate = 180, color = {gray!80, gray!20, gray!40, gray!60}]
	{48/Professionals,  9/Hedonists, 13/Indifferents, 30/Optimiser}
	\end{tikzpicture}
	\caption{Result of the cluster analysis: Four user groups}
	\label{fig:cluster} % \label has to be placed AFTER \caption to produce correct cross-references.
\end{figure}


\textbf{"Professionals" (48 \%):}
The Professionals are the biggest group. The members of this group are technically competent and interested in topics concerning energy.

The main characteristics are:
\begin{itemize}
	\item High proportion of persons having an abitur or university graduates
	\item Highest proportion of people in managerial positions, a quarter works (also) at home
	\item All household sizes (also households with children)
	\item Knowledge about energy
	\item High technical competence and interest in Technology. (Experience with home automation, a quarter has programming skills)
	\item Interested in sustainability
	\item Use of media or Internet is primarily for professional purposes
\end{itemize}

Typical segment behavior regarding home equipment:
 \begin{itemize}
 	\item "Reasonable" use of hot water ("I do not shower longer than necessary")
 	\item "Reasonable" use of lighting ("I turn down the light when I leave a room")
 	\item Make use of the "ECO-Button" (installed tool in the apartments of D12 which helps to save energy) when leaving the apartment
 \end{itemize}

Due to their technical expertise, their experience with home automation and their interest in energy issues they are the most appropriate target group for home automation and mobile application solutions. Rationally justified explanations and instructions for use meet their information style. Professionals also expect more detailed information in individual offers such as energy feedback.

\textbf{"Optimizer" (30 \%):}

The second largest segment comprises people who primarily aim to optimize energy costs. Optimizer have little knowledge about energy and are no technophiles.

The main characteristics are:
\begin{itemize}
	\item High proportion of persons having an abitur or university graduates
	\item Highest proportion of people in managerial positions
	\item More women
	\item All household sizes (also households with children)
	\item Interested in sustainability	
	\item Little to no knowledge about home automation
	\item no technophiles
	\item Use of media or Internet is not very noticeable
\end{itemize}

Typical segment behavior regarding home equipment:
\begin{itemize}
	\item Prefer to air manually rather than to make use of the automatic ventilation system
	\item A quarter never uses the "ECO-Button"
\end{itemize}

The use of the home furnishings indicates a poor understanding of their usability or less time of interaction with them. Due to their much lower competence in energy and technology compared to the professionals, the planned solutions and measures should focus very strongly on the following points:

\begin{itemize}
	\item Clear and concrete instructions for behavior, for example in the form of energy-saving tips or concrete, close to reality explanations and concrete benefits.
	\item Avoid technical language in communication and use personalized examples.
	\item Reduce energy feedback to essential information. Optimizer do not need detailed explanations. 
	\item Enable trouble shooting: Optimizer want a quick solution to an energy problem, as they do not want to spend lot of time on energy topics.
\end{itemize}

\textbf{"Indifferents" (13 \%):}

The "Indifferents" have low competence in energy and technology and no interest in energy topics or sustainability.

The main characteristics are:
\begin{itemize}
	\item Young segment
	\item High proportion of Non-workers
	\item No interest in sustainability
	\item Low technical competence (no experience with home automation)
	\item Information research and streaming is above average
\end{itemize}

Typical segment behavior regarding home equipment:
\begin{itemize}
	\item Hedonistic use of hot water: They enjoy taking long showers and baths
	\item Smallest number on different device types
	\item Little satisfaction with the provided air ventilation
\end{itemize}

The "Indifferents" have low interest in the research topic and it's solution in general. To address this group with the necessary knowledge and to awaken their interest for energy and sustainability, a bigger effort has to be done than for the above groups. A typical representative of this group is a person who has just moved out from the parental home and who now has to organize the household on his/her own and to develop independence.

\textbf{"Hedonists" (9 \%):}

The "Hedonists" are technical competent but are indifferent to energy and sustainability topics.

The main characteristics are:
\begin{itemize}
	\item Young segment
	\item More mens, more single households
	\item Technical competent and partly with programming skills
	\item Intensive use of mobile Apps and Internet
	\item Hedonistic use of gaming and social media
\end{itemize}

Typical segment behavior regarding home equipment:
\begin{itemize}
	\item Highest number on different device types
	\item Carefree use of lighting and hedonistic use of hot water
	\item Frequent use of "ECO-Button"
	\item High satisfaction with the provided air ventilation
	\item Weak identification with Seestadt
\end{itemize}

The youngest segment has good preconditions to make a good use of mobile application with feedback of their energy use. Nevertheless, the motivation to deal with energy topic is rather low. The hedonistic lifestyle with its strong convenience and comfort orientation is in the foreground. Despite the high usage of apps it may be difficult to win them around for energy feedback. The comfort gain is of great relevance.

\section{Usability Tests} mobile context
In order to avoid distorting of the research results the graphical user interface will be tested empirically with 4-5 usability tests, that means the usability is accessed by testing the interface with real users \cite{nielsen1994usability}.

\section{User study}
The design of the user study will follow the seminal guidelines for conducting case study research in software engineering as proposed by Runeson et al. \cite{runeson2012case}. The target group will consist of at least one user for each type of energy user. The study protocol will follow the check-lists for reading and reviewing case studies from H\"ost and Runeson \cite{host2007checklists}.

\section{Evaluation}
In this step the developed mobile app will be empirically evaluated against a valuation model in a user study to identify the success of the research. The evaluation model comprises of numerous Key Performance Indicators (KPIs). An extraction of these KPIs is listed in the following:
\begin{enumerate}
	\item More than 50 \% of all the users using the app state that the possibility of switching between different ways to display the information is useful
	\item More than 50 \% state, that they are more aware of what to do to avoid CO2 than before using the app
	\item More than 50 \% of the users state that they understand and get a feeling of how much CO2 they are producing
\end{enumerate}


Carbon dioxide awareness

\cite{mohammadmoradieffectiveness}

\section{Existing approaches}

\section{Comparison of existing approaches}
In this step, the market and competition analysis which was done when the problem arose will be done in more depth. The questions that shall be answered in this steps are the following.
\begin{itemize}
	\item Which applications are there within the topics of energy saving and CO2 awareness?
	\item Which approaches and visualizations do these applications make use of to increase awareness?
	\item How do these applications handle the users' level of education concerning energy units of measurements, such as kWh?
\end{itemize}

\section{Serious games}

The Energy Piggy Bank - A Serious Game for Energy Conservation

Serious games are games that educate, train, and inform

Serious games are gaining importance recently. These games aim at behavior change and education.

Hedin et al. \cite{Bjorn1165339} describe a serious game that shall help people learn more about their energy consumption. They designed the game according to the taxonomy of Bartles Player Types that constitute of four Types having different motivation for playing games.


They also evaluated the behaviour 

self-assessed future behaviour change 

The outcome of the work is a strong correlation between self-assessed future behavior change and perceived value/usevulness of the application independent of the player type.

Bartle Player Types

Serious games have attracted much attention recently and are used to in an engaging way support for example education and behavior change. In this paper, we present a serious game designed for helping people learn about their own energy consumption and to support behavior change towards more sustainable energy habits. We have designed the game for all the four Bartle Player Types, a taxonomy used to identify different motivations for playing games. Engagement of the participants has been evaluated using the Intrinsic Motivation Inventory, and we have measured self-assessed future behavior change. We found a statistically significant positive correlation between self-assessed future behavior change and perceived value/usefulness of the application independent of player type. Our study indicates the player type “Achievers” might perform better using this type of application and find it more enjoyable, but that it can be useful for learning energy conserving behavior independent of player type


\section{Persuasive System}
Tailoring and personalizing the content to the potential needs, interests, usage context or other factors is outlined by \cite{oinas2009persuasive} in the context of a Persuasive System. They studied how a persuasive system must be designed with tailored and personalized content to maximize the change in the user's behaviour. Although the outcome on the behaviour change is not relevant, the findings on the tailor aspects are highly interesting for the proposed thesis.


%The process of reconnecting is simply to stop. Stopping meditation 6 times a day for 10 seconds. Experiencing as a felt reality: Divinity
%Whatever you want to experience in your life: Be the source of experience for that for one another


