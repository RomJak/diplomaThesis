\chapter{Catalogue of Design Guidelines}

In this chapter the guidelines are listed to provide a compact catalogue. They are kept brief to enable a good overview of all the guidelines. The effect of each guideline is stated and the advantages and trade-offs are described separately. The issue coming along with the guideline gives further information for the designer and the background describes why the guideline has been derived.

\textbf{Design Guidelines}



\minitoc% Creating an actual minitoc

\section*{blubb}

\section*{foo}

\begin{table}
	\renewcommand{\arraystretch}{2}
	\textbf{Guideline 1:} Adapt navigation drawer to requirements of user type
	\newline

	\begin{tabular}{|p{2cm}p{10cm}|}
	
		\hline 
		Description & asaf fasfd asd fasdfasfasfdasfdasf asfdasd as \\ 
		\hline 
		Effect &  asdfa a sdf asdfasfd as afsf as fasdfas\\ 
		\hline 
		Upside &   asdf asfas\\ 
		\hline 
		Downside &  asfasd  \\ 
		\hline 
		Issues &  a sdfas \\ 
		\hline 
		Background & The paper prototype session with the professionals has shown that Professionals primarily use the app for monitoring their consumption rate, therefore the Dashboard is the first menu item. The main motivation for Optimizer is to save money with the app. Indifferents primarily use the app because it makes fun which means that their main motivation is supported by the gamification approach. Hedonists' are especially motivated to use the app, when their drive for programming projects is picked up. The gamification approach and the projects are shown in the first menu item, for that reason the Dashboard is on second place for Hedonists and Indifferents. The following user stories were tested in the paper prototype session with the according user types and were proven to appropriate:
		\begin{itemize}
			\item As a Professional I primarily use the app to monitor my consumption rate.
			\item As an Optimizer I primarily use the app to save money.
			\item As an Indifferent I primarily use the app for fun.
			\item As a Hedonist I primarily use the app to manage my home automation gadgets.
		\end{itemize}
		\\ 
		\hline
		
	\end{tabular}
	\renewcommand{\arraystretch}{1}
\end{table}




Sort the items of the navigation drawer according to the motivation of a user type. An effect would be that a user can quickly interact with the app as the preferred items are on top. The sorting reduces the time a user has to search for his/her primary task. However, the different sorting can be irritating when a user compares the app to a user who is a different type of energy user and therefore has another look of the app.

\textbf{Evidence} \quad 

\textbf{Description}: Sort the items of the navigation drawer according to the motivation of a user type

\textbf{Effect}: A user type can quickly interact with the app as the preferred items are on top. The sorting reduces the time a user has to search for his/her primary task.

\textbf{Upside}: The time a user searches for a special item is reduced.

\textbf{Downside}: The different sorting can be irritating when a user compares the app to a user who is a different type of energy user and therefore has another look of the app.

\textbf{Issues}: The

Background: The paper prototype session with the professionals has shown that Professionals primarily use the app for monitoring their consumption rate, therefore the Dashboard is on the first. The main motivation of Optimizer is to save money with the app. Indifferents primarily use the app because it makes fun which means that they would use it if the gamification approach is in front. Indifferents are interested in energy saving tips more if they can success in the game
Hedonists' primary use case for the app is home automation. For the Indifferents and the Hedonists their main motivation is presented under the first menu item.