\chapter{Suggested solution}
This chapter 


\section{Questionnaire for user segmentation}

In the State of the Art chapter the User Segmentation \ref{chap:usersegmentation} from Smart Cities Demo Aspern on which this thesis builds up upon is explained. With regard to the Paper Prototyping session we need users that clearly can be allocated to a user segment. To find users for each user segment we created a questionnaire. The user survey leaned on to the first questionnaire of the quantitative study of Aspern Smart City Research. The original questionnaire of ASCR comprised of 48 questions. The factor analysis of the returned questionnaire identified the four dimensions: Comfort-centred, Technology-centred, Data Sensibility and Living in Seestadt. The following cluster analysis found out four segments. For the definition of the segments only two of the four factors were relevant for describing the characteristics of a user group. For our study we focused on these two factors which are the comfort and the technology orientation. So, we took all the questions of the original questionnaire which answers were identified by the cluster analysis to be significant for the user segmentation. Out of the 48 questions of the original questionnaire ten were relevant for allocating a user to a user segment.

For creating and sending the survey we used Google Forms. As the motivation to find subjects who complete a survey increases as the question difficulty increases \cite{andrews2007conducting}, our questionnaire only comprised of ten questions and the average time for answering the whole questionnaire only took one minute. We sent the questionnaire to 57 people, trying to have a good distribution of different ages, educational levels, jobs and interests. 41 questionnaires were returned.

For evaluating the response we used Google Spreadsheet and Microsoft Excel. The answers of each person was evaluated against the characteristics of each of the four user segments. Of course not every user could easily be assigned to exactly one user segment. For each user the correspondence to each of the four user segments was calculated and expressed in percent. The ones who had a clear correspondence of more than 50 \% to one user type were chosen as test users for the paper prototyping, the usability tests and the user study later on. So at least one user for each user type was chosen. 

Given the answer of the first research questions and the results of the survey we can give a conclusive answer to the following research question:

\textbf{RQ 2: Which criteria do questions have to meet, that shall identify the type of a user?}

The questions shall concern the main characteristics of every user type regarding the factors technology and comfort. This means the survey should include questions that:

\begin{itemize}
	\item check the interest in energy
	\item investigate the knowledge of the consumption of electric devices
	\item detect the importance of saving at energy costs
	\item determine whether a user programs from time to time
	\item find out if home automation possibilities are used
	\item examine the pattern of showering or taking a bath
	\item check the behaviour of switching out the light when leaving a room
	\item ask if the light is sometimes forgotten to be switched of when leaving the apartment
	\item ask for the preferred use of lighting 	
\end{itemize}

Additionally, the questions shall be short, comprehensive and easy to answer, as mentioned in~\ref{sec:survey}. The whole questionnaire can be found in ~\nameref{chap:appendixA}.

\section{Recommendations for improvement for the ASCR App}

Insert screens in here and short description

\section{Paper Prototypes of the User Interfaces}
In order to elicit the requirements for the graphical user interfaces and to find usability mistakes we make use of Paper Prototyping. As described before in \ref{section:paperprototyping} we follow the Step-by-Step guide of Arnowitz et al. \cite{arnowitz2010effective} to create a Paper Prototype.

\begin{enumerate}
	\item \textbf{Create scenario}.
	
	The main goal of the mobile application that shall be developed is to give feedback about how much electricity, water and heating a user consumed, how much carbon-dioxide was produced and how the values can be made better. The screens shall be adapted to the user type to make the information and tips more attractive to a user's interests.
	
	\subitem{\textbf{Professionals}}: The user study revealed that Professionals have high interest in energy issues. As they are deep into the topic and prefer more detailed information in individual offers they can have a look on the energy consumption on a very detailed level, such as a consumption rate on a granularity of minutes.
 
	\begin{itemize}
		\item Task: Have a look on your consumption rate of the last week/month/year
	\end{itemize}
	
	Professionals should have a possibility within the application to compare energy consumptions of different time intervals.
	
	\begin{itemize}
		\item Task: Compare your consumption rate of the last week with the consumption rate of the same week one year earlier.
	\end{itemize}

	Professionals also like to compare themselves to others. Comparing his or her average energy consumption to others should also be possible.
	
	\begin{itemize}
		\item Task: Compare your consumption rate of the last week with the average consumption rate of your neighbours
	\end{itemize}
	
	Professionals also like rationally justified explanations and instructions for use. Notifications on a daily basis give tips on saving energy or CO2, give concrete instructions for use and provides deeper information in Energy topics.
	
	\begin{itemize}
		\item Task: Find tips on how to save more energy
	\end{itemize}
	
	\subitem{\textbf{Optimizer}}: Optimizer primarily aim at optimizing energy costs, so the app should give easy to find tips on how to save energy and therefore costs. Optimizers prefer less time of interaction. As Optimizer rather like unclear instructions, the notifications on a daily basis also should give concrete instructions on how to save energy or CO2.
	
	\begin{itemize}
		\item Task: Find out what to do to save costs for electricity
	\end{itemize}
	
	 Professionals like to know the concrete benefits of a certain behaviour change. The explanations shall be as close to reality as possible and technical language shall be avoided. The energy feedback is reduced to essential information and the detailed graphs for energy consumption that the Professionals get are not visible at a first glance for an Optimizer. The saved costs after a behaviour change shall also be visible to provide some kind of reward for the new habits.
	 
	 \begin{itemize}
	 	\item Task: Find out how much you have saved in the last week
	 \end{itemize}
	
	For Optimizer also trouble shooting shall be easily accessible, in order to reduce the time they are spending with the application and not to loose them on the way.
	
	\begin{itemize}
		\item Task: Report a problem
	\end{itemize}
	
	\subitem{\textbf{Indifferents}}:
	The Indifferents have low interest in energy topics in general, so the main requirement of the application for this type of user is in the first run to sensitize them for the topic, to raise awareness and to make electricity and CO2 saving appealing to them. 
	
	To awaken their interest for energy and sustainability a gamification approach will be used. For opening the application once a day the user earns points. Points are also earned for clicking on notifications and reading the article. Tips for saving energy or CO2 should not concern longer usage of laptops or entertainment screens, as streaming and use of social media is an important leisure activity for Indifferents.
	
	\begin{itemize}
		\item Task: Earn points by interacting with the App
	\end{itemize}
	
	\textit{Hedonists}:
	The youngest segment, the Hedonists, are keen on developing technical solutions. The interest in technology can be used to give instructions for programming technical devices and using home automation. The primary motive for the Hedonists is not to save energy but the interest in technology. This will be considered in the notifications and tips of the day. The hedonistic lifestyle with its strong convenience and comfort orientation is in the foreground.
	
	For a hedonist the comfort gain is of great relevance. Programming and establishing home automation aspects is a great interface between the aim of saving energy and the affinity of technology.
	
	
\end{enumerate}

\section{Design Guidelines for tailoring interfaces to user segments}

Guideline 1: Adapt navigation drawer to requirements of user type

Description: Sort the items of the navigation drawer differently and adapt them for each user type

Effect

Upside

Downside

Issues

Background

Guideline 2: Adapt navigation drawer to requirements of user type
Guideline 3: Adapt navigation drawer to requirements of user type
Guideline 4: Use Euro rather than kWh as measurement unit for Optimizer and Indifferents
Present primarily saving possibilities for Optimizer
Optimizer primarily use the app for saving purposes

Use concrete insturction for Optimizer and avoid detailed information
Optimizer prefer concrete instructions to detailed information

Present a hotline for trouble shooting for optimizer
Optimizer prefer calling when an error occurs


Optimizer prefer concrete instructions to control possibilities
Indifferents use the app if it makes fun
Indifferents are interested in energy saving tips more if they can success in the game
Indifferents


 




\begin{figure}[h]
	\centering
	\begin{subfigure}[b]{0.24\columnwidth}
		\centering
		\includegraphics[width=\textwidth]{screens/drawer_1}
		\subcaption{Professional}
		\label{fig:drawer:professional}
	\end{subfigure}
	\begin{subfigure}[b]{0.24\columnwidth}
		\centering
		\includegraphics[width=\textwidth]{screens/drawer_2}
		\subcaption{Optimizer}
		\label{fig:drawer:optimizer}
	\end{subfigure}
	\begin{subfigure}[b]{0.24\columnwidth}
		\centering
		\includegraphics[width=\textwidth]{screens/drawer_3}
		\subcaption{Indifferent}
		\label{fig:drawer:indifferent}
	\end{subfigure}
	\begin{subfigure}[b]{0.24\columnwidth}
		\centering
		\includegraphics[width=\textwidth]{screens/drawer_4}
		\subcaption{Hedonist}
		\label{fig:drawer:hedonist}
	\end{subfigure}
	\caption{The proposed screen for the navigation drawer}
	\label{fig:drawer} % \label has to be placed AFTER \caption (or \subcaption) to produce correct cross-references.
\end{figure}

\begin{figure}[h]
	\centering
	\begin{subfigure}[b]{0.24\columnwidth}
		\centering
		\includegraphics[width=\textwidth]{screens/dashboard_14}
		\subcaption{Professional and Hedonist}
		\label{fig:dasboard:professional}
	\end{subfigure}
	\begin{subfigure}[b]{0.24\columnwidth}
		\centering
		\includegraphics[width=\textwidth]{screens/dashboard_23}
		\subcaption{Optimizer and Indifferent}
		\label{fig:dashboard:optimizer}
	\end{subfigure}
	\caption{The proposed screen for the dashboard}
	\label{fig:dashboard} % \label has to be placed AFTER \caption (or \subcaption) to produce correct cross-references.
\end{figure}

\begin{figure}[h]
	\centering
	\begin{subfigure}[b]{0.24\columnwidth}
		\centering
		\includegraphics[width=\textwidth]{screens/aktuelles_14}
		\subcaption{Professional and Hedonist}
		\label{fig:aktuelles:professional}
	\end{subfigure}
	\begin{subfigure}[b]{0.24\columnwidth}
		\centering
		\includegraphics[width=\textwidth]{screens/aktuelles_23}
		\subcaption{Optimizer and Indifferent}
		\label{fig:aktuelles:optimizer}
	\end{subfigure}
	\caption{The proposed screen for the navigation drawer}
	\label{fig:aktuelles} % \label has to be placed AFTER \caption (or \subcaption) to produce correct cross-references.
\end{figure}

\begin{figure}[h]
	\centering
	\begin{subfigure}[b]{0.24\columnwidth}
		\centering
		\includegraphics[width=\textwidth]{screens/Statistik_1234}
		\subcaption{Professional and Hedonist}
		\label{fig:statistik:professional}
	\end{subfigure}
	\begin{subfigure}[b]{0.24\columnwidth}
		\centering
		\includegraphics[width=\textwidth]{screens/Statistik_Vergleich}
		\subcaption{Optimizer and Indifferent}
		\label{fig:statistik:optimizer}
	\end{subfigure}
	\caption{The proposed screens for statistics}
	\label{fig:statistik} % \label has to be placed AFTER \caption (or \subcaption) to produce correct cross-references.
\end{figure}

\begin{figure}[h]
	\centering
	\begin{subfigure}[b]{0.24\columnwidth}
		\centering
		\includegraphics[width=\textwidth]{screens/tipp_1}
		\subcaption{Professional}
		\label{fig:tipps:professional}
	\end{subfigure}
	\begin{subfigure}[b]{0.24\columnwidth}
		\centering
		\includegraphics[width=\textwidth]{screens/tipp_2}
		\subcaption{Optimizer}
		\label{fig:tipps:optimizer}
	\end{subfigure}
	\begin{subfigure}[b]{0.24\columnwidth}
		\centering
		\includegraphics[width=\textwidth]{screens/tipp_3}
		\subcaption{Indifferent}
		\label{fig:tipps:indifferent}
	\end{subfigure}
	\begin{subfigure}[b]{0.24\columnwidth}
		\centering
		\includegraphics[width=\textwidth]{screens/tipp_4}
		\subcaption{Hedonist}
		\label{fig:tipps:hedonist}
	\end{subfigure}
	\caption{The paper prototype screen for energy saving tips}
	\label{fig:tipps} % \label has to be placed AFTER \caption (or \subcaption) to produce correct cross-references.
\end{figure}

\begin{figure}[h]
	\centering
	\begin{subfigure}[b]{0.24\columnwidth}
		\centering
		\includegraphics[width=\textwidth]{screens/Sparen_1}
		\subcaption{Professional}
		\label{fig:sparen:professional}
	\end{subfigure}
	\begin{subfigure}[b]{0.24\columnwidth}
		\centering
		\includegraphics[width=\textwidth]{screens/Sparen_2}
		\subcaption{Optimizer}
		\label{fig:sparen:optimizer}
	\end{subfigure}
	\begin{subfigure}[b]{0.24\columnwidth}
		\centering
		\includegraphics[width=\textwidth]{screens/Sparen_3}
		\subcaption{Indifferent}
		\label{fig:sparen:indifferent}
	\end{subfigure}
	\begin{subfigure}[b]{0.24\columnwidth}
		\centering
		\includegraphics[width=\textwidth]{screens/Sparen_4}
		\subcaption{Hedonist}
		\label{fig:sparen:hedonist}
	\end{subfigure}
	\caption{The paper prototype screen for saving}
	\label{fig:sparen} % \label has to be placed AFTER \caption (or \subcaption) to produce correct cross-references.
\end{figure}

\begin{figure}[h]
	\centering
	\begin{subfigure}[b]{0.24\columnwidth}
		\centering
		\includegraphics[width=\textwidth]{screens/Trophaensammlung}
		\subcaption{Professional and Hedonist}
		\label{fig:trophaen}
	\end{subfigure}
	\begin{subfigure}[b]{0.24\columnwidth}
		\centering
		\includegraphics[width=\textwidth]{screens/Zonenkalender}
		\subcaption{Optimizer and Indifferent}
		\label{fig:zonen:optimizer}
	\end{subfigure}
	\caption{The proposed screens for statistics}
	\label{fig:kalender-} % \label has to be placed AFTER \caption (or \subcaption) to produce correct cross-references.
\end{figure}
