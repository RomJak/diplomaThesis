\chapter{Questionnaire for user segmentation}

In the State of the Art chapter the User Segmentation \ref{chap:usersegmentation} from Smart Cities Demo Aspern on which this thesis builds up upon is explained. With regard to the Paper Prototyping session we need users that clearly can be allocated to a user segment. To find users for each user segment we created a questionnaire. This questionnaire answers the second research question:

\textbf{RQ 2: Which criteria do questions have to meet, that shall identify the type of a user?}


The user survey leaned on to the first questionnaire of the quantitative study of Aspern Smart City Research. The original questionnaire of ASCR comprised of 48 questions. The factor analysis of the returned questionnaire identified the four dimensions: Comfort-centred, Technology-centred, Data Sensibility and Living in Seestadt. The following cluster analysis found out four segments. For the definition of the segments only two of the four factors were relevant for describing the characteristics of a user group. For our study we focused on these two factors which are the comfort and the technology orientation. So, we took all the questions of the original questionnaire which answers were identified by the cluster analysis to be significant for the user segmentation. Out of the 48 questions of the original questionnaire ten were relevant for allocating a user to a user segment.

For creating and sending the survey we used Google Forms. As the motivation to find subjects who complete a survey increases as the question difficulty increases \cite{andrews2007conducting}, our questionnaire only comprised of ten questions and the average time for answering the whole questionnaire only took one minute. We sent the questionnaire to 57 people, trying to have a good distribution of different ages, educational levels, jobs and interests. 41 questionnaires were returned.

For evaluating the response we used Google Spreadsheet and Microsoft Excel. The answers of each person was evaluated against the characteristics of each of the four user segments. Of course not every user could easily be assigned to exactly one user segment. For each user the correspondence to each of the four user segments was calculated and expressed in percent. The ones who had a clear correspondence of more than 50 \% to one user type were chosen as test users for the paper prototyping, the usability tests and the user study later on. So at least one user for each user type was chosen. 

Given the answer of the first research questions and the results of the survey we can give a conclusive answer to RQ2. The questions shall concern the main characteristics of every user type regarding the factors technology and comfort. This means the survey should include questions that:

\begin{itemize}
	\item check the interest in energy
	\item investigate the knowledge of the consumption of electric devices
	\item detect the importance of saving at energy costs
	\item determine whether a user programs from time to time
	\item find out if home automation possibilities are used
	\item examine the pattern of showering or taking a bath
	\item check the behaviour of switching out the light when leaving a room
	\item ask if the light is sometimes forgotten to be switched of when leaving the apartment
	\item ask for the preferred use of lighting 	
\end{itemize}

Additionally, the questions shall be short, comprehensive and easy to answer, as mentioned in~\ref{sec:survey}. The whole questionnaire can be found in ~\nameref{chap:appendixA}.


