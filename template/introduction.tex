\chapter{Introduction}

In the age of social media, where information is tailored to users' interests, preferences and state of education, the question arises how to integrate this phenomenon into common mobile applications. Especially when it comes to education and behaviour change an adaption of the user interface to various requirements might be beneficial.

This thesis investigates whether tailoring the interface of a mobile application to a user's state of knowledge has effects on the usefulness of the application. The state of knowledge and the needs of the users are gathered into user groups, in order to limit the amount of possibilities.


\section{Motivating Scenario}

This thesis is written in cooperation with Siemens AG Austria, within a research project that deals with the Seestadt in Aspern. The Seestadt is one of the biggest city development projects in Europe \footnote{https://www.aspern-seestadt.at/ Accessed 10.01.2018}. The Aspern Smart City Research GmbH \& Co KG \footnote{http://www.ascr.at/ Accessed 10.01.2018} (ASCR) is an exclusive technology partner of Siemens AG. The Aspern project has the overall goal of finding smarter solutions for energy consumption with the help of smart grids, power supplies, building systems, intelligent power grids, and information
and communication technologies (ICT) that should interact in an optimal manner. The ASCR infrastructure manages the data coming from smart grids and smart buildings such as temperature, energy consumption, water consumption, power demand as well as external data sources such as weather, city events, energy market, traffic reports etc. \cite{parreira2015role} In total 1.5 million values are measured per day. To create something useful out of this amount of big data is a big task. \footnote{http://www.report.at/index.php/energie/wirtschaft-a-politik/item/91884-lebendes-stadtlabor Accessed 10.01.2018}

Take, for example, an application that informs you about your electricity consumption. What can be assumed, is that the user wants an easy-to-use application which shows the power consumption in an understandable way. The problem that we observed is that the majority of users lack the feeling for the size of one kilowatt hour. The same can be witnessed when it comes to CO2 emission. The unit of kilograms of CO2 is an information that mostly only experts can grasp and can relate to.

\section{Problem Statement}

In the field of software development the interaction with the user is important, including the consideration of a user's knowledge. Numerous applications aim at motivating the user to save energy or CO2 but neglect the incomprehensibility of units of measurements one does not deal with on a daily basis. The sense of trying to motivate the user to save energy by displaying the electricity consumption in kilowatt hours, might have less impact than setting it at least in relation to an average consumption of electricity or even visualizing it with a gamification approach. On the other hand, for someone who is easy on these types of measurement a visualization with colours or graphs might be too much.

So, the problem we are facing is to develop a mobile application that is beneficial for all types of users, starting from users who do not have a feeling for kilowatt hours or kilograms of CO2 up to users having a great affinity for electricity and carbon-dioxide emission.

To address this bandwidth of user knowledge and visualization possibilities, this thesis investigates the usefulness of tailoring a mobile application to a users knowledge. Furthermore design principles and criteria that shall help front-end developer, usability engineers as well as software architects to develop applications customized to a users level of knowledge shall be investigated.

We evaluate different types of users and their preferred way of gathering information. Ranging from the ones who show only interest in their overall behaviour, meaning if they are better or worse than the average, over others, who want to know their power consumption more detailed but still can't grasp the measurement of one kilowatt hour, up to users, who are deep into the topic and are keen on extensive figures.

\section{Aim of the work} 
The overall goal of this thesis is to identify the benefits or even drawbacks of providing a user interface in a mobile application with various possibilities of presenting information to switch between. We want to investigate if a user makes use of different visualizations or the presented way is excepted and therefore an adaptation of the user to the application takes place.\\
This thesis contributes 
(1) a prototype of a mobile application aiming at increasing CO2 awareness with the help of customized visualizations and
(2) a catalogue of criteria of design principles for tailoring visualizations containing information of consumption data.

The central research question is the following:

\textbf{What are the effects on knowledge acquisition in mobile applications when providing visualizations tailored to users' knowledge?}
\\\\
The central research question can be answered after having found a solution to the sub-questions:\\\\
\textbf{(a) What are the characteristics of a user group with the same state of knowledge?}
In order to answer this research question we first conduct a literature review in the area, followed by a user survey detecting the state of knowledge in the field of electricity units of measurements, i.e. the size of one kilowatt hour, one kilogram of CO2. These findings will help in identifying groups and their characteristics.\\\\
\textbf{(b) Which criteria do questions have to meet, that shall identify the type of a user?}
The findings of the sub-research question (a) will have an influence on the questionnaire needed for defining which group a user can be assigned to. This questionnaire will be the first contact point in the mobile application.
\\\\
\textbf{(c) What are the design possibilities when it comes to tailoring interfaces to a users' state of knowledge in the scope of electricity consumption data?}
This question can be answered by conducting a literature review and considering the characteristics of a user group.
\\\\
\textbf{(d) Do the characteristics of user groups correlate with the users' preferred type of visualization?}
The results elicited for research question a) are the foundation for defining the correlation between groups of users and their preferred type of visualization. Assuming the favourite type of visualization is the most used one, allows to identify the preferred type of visualization by analysing the log files.
\\\\
\textbf{(e) Does a user switch between various screens showing the same information represented in different ways?}
We answer this by looking at the log files and also by observing the interaction with the mobile application in the usability tests.

%---------------------------------------------------------------------------
\section{Methodological Approach}
%---------------------------------------------------------------------------

In order to fulfill the research questions the methodological approach comprises the following steps:
\begin{enumerate}
	\item \textbf{Literature Review} \\
	The first step is to dive into the topic of usability engineering, especially different forms of visualizations and graphical user interfaces in the scope of mobile applications. That implies a research about paper prototyping, usability testing in the mobile context as well as user classification and carbon dioxide awareness. The goal is to get an insight of all relevant aspects which will serve as foundation for the following steps.
	
	\item \textbf{Comparative analysis of alternatives and comparison of existing approaches} \\
	In this step, the market and competition analysis which was done when the problem arose will be done in more depth. The questions that shall be answered in this steps are the following.
	\begin{itemize}
		\item Which applications are there within the topics of energy saving and CO2 awareness?
		\item Which approaches and visualizations do these applications make use of to increase awareness?
		\item How do these applications handle the users' level of education concerning energy units of measurements, such as kWh?
	\end{itemize}
	
	\item \textbf{Elicitation of requirements with Paper Prototyping} \\
	The second step is to do "Paper Prototyping" in order to elicit the requirements for the graphical user interfaces and overall for the CO2 awareness app. According to \cite{lancaster2004paper} the numerous benefits of early usability studies are vastly superior. It may seem low-tech, but conducting usability tests at this step show what users really expect on a quite detailed level which gives maximum feedback for minimum effort \cite{weiss2003handheld}.
	
	At first a group of people containing at least one user for each user type will be put together.  Next, hand-sketched drafts will be drawn, showing the app with menus, dialog boxes, notifications and buttons. Then, different tasks that can be done with the app shall be defined. These tasks are then conducted by the users. The feedback from the users show what they expect from the app which is of great value for the implementation later on \cite{snyder2003paper}.
	
	\item \textbf{Architectural Design of the CO2 awareness mobile application} \\
	The insights from the previous steps will influence the architecture and the designs of the CO2 awareness mobile application. With focus on design and usability an architectural design will be developed including a development plan. At this step, the different data resources for the computation of the personal CO2 emission, such as power consumption, water consumption, nutrition lifestyle, transportation habits, size of the living space, place of living, family situation etc. must be considered.
	
	The app shall be usable for all users but will be particularly useful for inhabitants of the Aspern Seestadt in Vienna, as we have a database for the dwellers of the student dorm, the schoolhouse and one residential building. This data comes from the Aspern Smart City Research\footnote{http://www.ascr.at/. Accessed 9.11.2017} (ASCR) project where Siemens plays an essential role in collaboration research. 
	
	\item \textbf{Technical Implementation of the CO2 awareness mobile application} \\
	According to the architecture description from step 4 the mobile application will be implemented using an agile software development process and a fully native approach targeting Android Devices.
	
	\item \textbf{Usability Tests} \\
	In order to avoid distorting of the research results the graphical user interface will be tested empirically with 4-5 usability tests, that means the usability is accessed by testing the interface with real users \cite{nielsen1994usability}.
	
	\item \textbf{User study} \\
	The design of the user study will follow the seminal guidelines for conducting case study research in software engineering as proposed by Runeson et al. \cite{runeson2012case}. The target group will consist of at least one user for each type of energy user. The study protocol will follow the check-lists for reading and reviewing case studies from H\"ost and Runeson \cite{host2007checklists}.
	
	\item \textbf{Evaluation} \\
	In this step the developed mobile app will be empirically evaluated against a valuation model in a user study to identify the success of the research. The evaluation model comprises of numerous Key Performance Indicators (KPIs). An extraction of these KPIs is listed in the following:
	\begin{enumerate}
		\item More than 50 \% of all the users using the app state that the possibility of switching between different ways to display the information is useful
		\item More than 50 \% state, that they are more aware of what to do to avoid CO2 than before using the app
		\item More than 50 \% of the users state that they understand and get a feeling of how much CO2 they are producing
	\end{enumerate}
	
	
\end{enumerate}


\section{Structure of the work} 
The remainder of this thesis is structured as follows:  Chapter 2 provides an overview of related work where the main approaches of tailoring user interfaces are discussed. This chapter is concluded by a comparison with the existing approaches.

%Subsequently in Chapter 3 the methodology is presented, where an overall architecture is explained apart from the concrete implementation. This is followed by an example use case, motivating the implementation of this thesis. Then, the conceptional architecture is explained, which is described in more detail in the Data Models and Design Methods section.
%
%Afterwards, in Chapter 4 the main work of this thesis, the implementation, is presented. Within this chapter implementation-specific details are discussed.
%
%Then, Chapter 5 evaluates the implementation in form of an example work flow and presents all the opportunities within the framework.
%
%Chapter 6 critically reflects and compares the implementation with related work and discusses open issues.
%
%This thesis is concluded in Chapter 7 with a summary and outlook on future work.

