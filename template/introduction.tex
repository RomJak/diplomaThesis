\chapter{Introduction}

In the age of social media, where information is tailored to users' interests, preferences and state of education, the question arises how to integrate this phenomenon into common mobile applications. Especially when the aim of the application is education or changing a user's behavior an adoption of the user interface to various requirements might be beneficial.

This thesis investigates whether tailoring the interface of a mobile application to a user's characteristics, preferences and state of knowledge has effects on the usefulness of the application. The state of knowledge and the needs of the users are gathered into user groups, in order to limit the amount of possibilities.


\section{Motivating Scenario}

This thesis is written in cooperation with Siemens AG Austria, within a research project that deals with the Seestadt in Aspern in Viennna. The Seestadt is one of the biggest city development projects in Europe \footnote{https://www.aspern-seestadt.at/ Accessed 10.01.2018}. The Aspern Smart City Research GmbH \& Co KG \footnote{http://www.ascr.at/ Accessed 10.01.2018} (ASCR) is an exclusive technology partner of Siemens AG. The Aspern project has the overall goal of finding smarter solutions for energy consumption with the help of smart grids, power supplies, building systems, intelligent power grids and communication technologies. Another side goal is an optimal interaction of all these components. The ASCR infrastructure manages the data coming from smart grids and smart buildings such as temperature, energy consumption, water consumption, power demand as well as external data sources such as weather, city events, energy market, traffic reports etc. \cite{parreira2015role}. In total 1.5 million values are measured per day \footnote{http://www.report.at/index.php/energie/wirtschaft-a-politik/item/91884-lebendes-stadtlabor Accessed 10.01.2018}. To create something useful out of this amount of data is a big task. 

Take, for example, an application that informs you about your electricity consumption. What can be assumed, is that the user wants an easy-to-use and beneficial application for the user. Users are different and so are the motives why someone uses an application, e.g. saving energy, monitoring consumption, pure interest etc. A company or a mobile application developer of course wants to develop an application that serves as much people as possible. But what to do when the target group is defined but consist of people with distinct interests and different level of knowledge?

 The problem that we observed is that the majority of users lack the feeling for the size of one kilowatt hour. The same can be witnessed when it comes to CO2 emission. The unit of kilograms of CO2 is an information that mostly only experts can grasp and can relate to.

\section{Problem Statement}

In the field of software development the interaction with the user is important, including the consideration of a user's knowledge. Numerous applications aim at motivating the user to save energy or CO2 but neglect the incomprehensibility of units of measurements one does not deal with on a daily basis. The sense of trying to motivate the user to save energy by displaying the electricity consumption in kilowatt hours, might have less impact than setting it at least in relation to an average consumption of electricity or even visualizing it with a playful approach. On the other hand, for someone who is easy on these types of measurement a visualization with colours or graphs might be too much.

So, the problem we are facing is to develop a mobile application that is beneficial for all types of users, starting from users who do not have a feeling for kilowatt hours or kilograms of CO2 and may not even be interested in energy topics up to users having a great affinity for electricity and carbon-dioxide emission.

To address this bandwidth of user knowledge and visualization possibilities, this thesis investigates the usefulness of tailoring a mobile application to a users knowledge. Furthermore design principles and criteria that shall help front-end developers, usability engineers as well as software architects to develop applications customized to a users level of knowledge shall be investigated.

\section{Aim of the work} 
The overall goal of this thesis is to identify different type of users, to evaluate existing applications and to analyse the benefits or even drawbacks of providing user interfaces in a mobile application tailored to a user type.\\
This thesis contributes 
(1) a questionnaire for identifying the energy type of a user 
(2) an evaluation of a mobile application in the field of consumption data and home automation
(3) paper prototypes for a mobile application with interfaces tailored to user types
(4) a catalogue of criteria of design principles for tailoring visualizations for eco-feedback

The central research question is the following:

\textbf{Are there benefits of providing visualizations tailored to user segments?}
\\\\
The central research question can be answered after having found a solution to the sub-questions:\\\\
\textbf{RQ 1: What are the characteristics of a user segment with the same energy consumption interests?}
In order to answer this research question we conduct a literature review in this area, where we want to find out different user types, the characteristics, the state of knowledge and the preferred way of interaction with them.\\\\
\textbf{RQ 2: Which criteria do questions have to meet, that shall identify the type of a user?}
The findings of RQ 1 will have an influence on the questionnaire needed for defining which group a user can be assigned to. The questionnaire shall be short and shall precisely identify the type of a user.
\\\\
\textbf{RQ 3: What are the design possibilities when it comes to tailoring interfaces to a user segment in the scope of electricity consumption data?}
This question can be answered by evaluating existing approaches of eco-feedback applications and analysing their approach of trying to change a user's behaviour. One application will be investigated in particular and will be the basis for the paper prototypes.
\\\\
\textbf{RQ 4: Do the characteristics of user groups correlate with the users' preferred type of visualization?}
The results elicited for RQ 1 are the foundation for defining the user groups. The outcome of RQ 3, the paper prototypes, will be presented to the test users found with the questionnaire of RQ 2. The correlation between the groups 
and their preferred type of visualization will be determined in workshops with the user types.


%---------------------------------------------------------------------------
\section{Methodological Approach}
%---------------------------------------------------------------------------

In order to answer the research questions the methodological approach comprises the following steps:
\begin{enumerate}
	\item \textbf{Literature Review} \\
	The first step is to dive into the topic of usability engineering, especially different forms of visualizations and graphical user interfaces in the scope of mobile applications. That implies a research about paper prototyping, usability testing in the mobile context as well as user classification and evaluation of user interfaces. The goal is to get an insight of all relevant aspects which will serve as foundation for the following steps and also to get a base that shall help at answering the research questions.
	
	\item \textbf{Comparative analysis of alternatives and comparison of existing approaches} \\
	In this step, the market and competition analysis which was done when the problem arose will be done in more depth. The questions that shall be answered in this steps are the following.
	\begin{itemize}
		\item Which applications are there within the topics of energy saving and CO2 awareness?
		\item Which approaches and visualizations do these applications make use of to present feedback?
		\item How do these applications tailor their visualizations to different requirements?
	\end{itemize}

	\item \textbf{Creating a questionnaire for user classification} \\
	In this step one contribution of this thesis is created, the questionnaire for the identification of the segment a user can be assigned to. This will be a short questionnaire that shall identify the correspondence to the main characteristics of a user segment such as interest in energy topics, typical usage patterns of consumption, technical competence etc.. The questions shall help to answer RQ 2 Then the questionnaire will be sent out in order to find at least one person for each user segment.
	
	\item \textbf{Design of Paper Prototypes} \\
	The second step is to create paper prototypes. The prototypes shall follow usability guidelines found out in the previous steps. The whole paper prototyping process will be close to the Step-by-Step guide for creating Paper Prototypes proposed by Arnowitz et al. \cite{arnowitz2010effective}. This includes first, the definition of the goal followed by identifying the tasks that users shall be able to do with the App. Next, hand-sketched drafts will be drawn, showing the application with menus, dialog boxes, notifications and buttons.
	
	\item \textbf{Elicitation of requirements with Paper Prototyping} \\
	The third step is to do the Paper Prototyping session in order to elicit the requirements for the graphical user interfaces and overall for the app. According to \cite{lancaster2004paper} the numerous benefits of early usability studies are vastly superior. It may seem low-tech, but conducting usability tests at this step show what users really expect on a quite detailed level which gives maximum feedback for minimum effort \cite{weiss2003handheld}.
	
	At first the people that could be clearly assigned to one user segment will be invited to a paper prototyping workshop. The workshops for each user segment will be held separately in order to avoid the distortion of results and to create a mutual independent outcome. The feedback from the users show what they expect from the app which is of great value for the further design of the app \cite{snyder2003paper} and for the following evaluation.
	
%	\item \textbf{Architectural Design of the CO2 awareness mobile application} \\
%	The insights from the previous steps will influence the architecture and the designs of the CO2 awareness mobile application. With focus on design and usability an architectural design will be developed including a development plan. At this step, the different data resources for the computation of the personal CO2 emission, such as power consumption, water consumption, nutrition lifestyle, transportation habits, size of the living space, place of living, family situation etc. must be considered.
%	
%	The app shall be usable for all users but will be particularly useful for inhabitants of the Aspern Seestadt in Vienna, as we have a database for the dwellers of the student dorm, the schoolhouse and one residential building. This data comes from the Aspern Smart City Research\footnote{http://www.ascr.at/. Accessed 9.11.2017} (ASCR) project where Siemens plays an essential role in collaboration research. 
		
%	\item \textbf{Usability Tests} \\
%	In order to avoid distorting of the research results the graphical user interface will be tested empirically with 4-5 usability tests, that means the usability is accessed by testing the interface with real users \cite{nielsen1994usability}.
%	
%	\item \textbf{User study} \\
%	The design of the user study will follow the seminal guidelines for conducting case study research in software engineering as proposed by Runeson et al. \cite{runeson2012case}. The target group will consist of at least one user for each type of energy user. The study protocol will follow the check-lists for reading and reviewing case studies from H\"ost and Runeson \cite{host2007checklists}.
	
	\item \textbf{Evaluation} \\
	In this step the mobile application from ASCR will be empirically evaluated against the outcomes of the paper prototyping session and its benefits and drawbacks will be defined. The outcomes of the paper prototype sessions will help answering RQ 4, which deals with the correlation of user characteristics and the preferred type of visualization 
	
	\item \textbf{Refinement of design principles catalogue} \\
	Finally, based on the findings of the evaluation a design principle catalogue will be created. The catalogue will comprise motivations of the user types, requirements definitions and example implementations from the evaluated paper prototypes.
	
\end{enumerate}


\section{Structure of the work} 
The remainder of this thesis is structured as follows:  Chapter 2 provides an overview of related work where existing approaches of tailoring user interfaces are discussed beginning with an overview of the foundation of interfaces, the usability. This chapter is concluded by a comparison of existing approaches.

Subsequently in Chapter 3 the methodology is presented, where the guidelines for the survey for user segmentation and a step-by-step guide for paper prototyping is explained and the approach for the definition of design guidelines is outlined.

Afterwards, in Chapter 4 the main work of this thesis, the paper prototypes, the prototyping workshops and the evaluation of the ASCR mobile application, is presented. Within this chapter implementation-specific details are discussed.

Chapter 5 critically reflects and compares the implementation with related work and discusses open issues.

This thesis is concluded in Chapter 6 with a summary and a discussion on future work.

